\subsection{Conjunto}
\noindent Un \underline{conjunto}, es una colección definida de objetos que comparten unas propiedades. Pueden ser finitos o infinitos.
\[
        A = \{a_1,a_2,... a_k\}
\]
\[
        B = \{b_1,b_2,... b_k\}
\]
\noindent\fbox{{\large Propiedades}} \par
\begin{itemize}
        \item \(A\bigcap B\) corresponde a la \underline{intersección}, y abarca a todos los elementos que comparten ambos conjuntos.
        \item \(A\bigcup B\) corresponde a la \underline{unión}, y abarca al total de elementos que contienen ambos conjuntos.
        \item \(\left | A \right |\) corresponde al \underline{cardinal}, y son el numero total de elementos que contiene el conjunto.
        \item \(A \subseteq  B\) indica que \(A\) es un subconjunto de \(B\).
        \item \(A\setminus B\) corresponde a los elementos en \(A\) no presentes en \(B\). Se suele escribir como \(\overline{A}\).
        \item \(\overline{\overline{A}} = A\)
        \item \(\overline{A\bigcup B} = \overline{A}\bigcap \overline{B}\)
        \item \(\overline{A\bigcap B} = \overline{A}\bigcup  \overline{B}\)
        \item \(A \bigcup \overline{A} = \O\)
\end{itemize}
\subsection{Principios}
\subsubsection{Principio de Adición}
\subsubsection{Principio de Producto / Ley del Producto}
\subsubsection{Principio de Inclusión y Exclusión}
\subsubsection{Principio de Distribución / Dirichlet}
\subsection{Combinatoria, casos}
\subsubsection{Variaciones}
\subsubsection{Permutaciones}
\subsubsection{Combinaciones}
\subsubsection{Emparejamientos y Desarreglos}
\subsection{Número Binómico}
\subsubsection{Triangulo de Pascal}
\subsubsection{Binomio de Newton}