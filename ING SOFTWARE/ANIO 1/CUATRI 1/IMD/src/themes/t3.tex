\subsection{Conjunto}
 Un \underline{conjunto}, es una colección definida de objetos que comparten unas propiedades. Pueden ser finitos o infinitos.
\[
        A = \{a_1,a_2,... a_k\}
\]
\[
        B = \{b_1,b_2,... b_k\}
\]
\fbox{{\large Propiedades}} \par
\begin{itemize}
        \item \(A\cap B\) corresponde a la \underline{intersección}, y abarca a todos los elementos que comparten ambos conjuntos.
        \item \(A\cup B\) corresponde a la \underline{unión}, y abarca al total de elementos que contienen ambos conjuntos.
        \item \(\left | A \right |\) corresponde al \underline{cardinal}, y son el numero total de elementos que contiene el conjunto.
        \item \(A \subseteq  B\) indica que \(A\) es un subconjunto de \(B\).
        \item \(A\setminus B\) corresponde a los elementos en \(A\) no presentes en \(B\). Se suele escribir como \(\overline{A}\).
        \item \(\overline{\overline{A}} = A\)
        \item \(\overline{A\cup B} = \overline{A}\cap \overline{B}\)
        \item \(\overline{A\cap B} = \overline{A}\cup  \overline{B}\)
        \item \(A \cup \overline{A} = \O\)
        \item \(\left | P(A) \right | = 2^{\left | A \right |}\) indica el tamaño máximo del conjunto.
\end{itemize}
\fbox{{\large Diagramas de Venn}} \par
\vspace{1cm}
\begin{itemize}
        \item[] \begin{venndiagram2sets}
                        \fillA \fillB
                \end{venndiagram2sets}
                Unión: \(\mathbf{A \cup B}\)

        \item[]       \begin{venndiagram2sets}
                        \fillACapB
                \end{venndiagram2sets}
                Intersección: \(\mathbf{A \cap B}\)
        \item[] \begin{venndiagram2sets}
                        \fillOnlyB
                \end{venndiagram2sets}
                Complemento de A: \(\mathbf{\overline{A} = B \setminus A}\)
        \item[] \begin{venndiagram2sets}
                        \fillOnlyA
                \end{venndiagram2sets}
                Complemento de B: \(\mathbf{\overline{B} = A \setminus B}\)
\end{itemize}
\subsection{Principios}
 Aquí veremos 4 principios que nos servirán para operar más adelante:
\subsubsection{Principio de Adición}
 Siempre y cuando dos  o más conjuntos no tengan ningún elemento en común, la cardinalidad de su intersección dará lugar a la suma de sus cardinalidades:
\[
        \boxed{A \cap B = \O  \hspace{2mm}\Rightarrow\hspace{2mm} \left | A \cap B \right | = \left | A \right | + \left | B \right |}
\]
\subsubsection{Principio de Producto / Ley del Producto}
 Si hay dos o más conjuntos finitos, en los que multipliques un conjunto con una cardinalidad menor o igual que el consecutivo, entonces el producto entre ambos conjuntos será el producto cartesiano:
\[
        \boxed{\left | A \times B \right | = \left | A  \right | \times \left | B  \right |}
\]
\subsubsection{Principio de Inclusión y Exclusión}
 Para obtener la cardinalidad de todos los elementos que abarcan dos o más conjuntos, su intersección, se debe de sumar la cardinalidad de todos los conjuntos a la cardinalidad de la intersección de cada 3 elementos, y restarle la intersección de cada par de conjuntos.
\\Aquí podemos ver el caso de dos conjuntos:
\[
        \boxed{\left | A \cup B \right | = \left | A \right | + \left | B \right | - \left | A \cap B \right |}
\]
 Aquí el de 3:
\[
        \boxed{\left | A \cup B \cup C \right | = \left | A \right | + \left | B \right |+ \left | C \right |-( \left | A \cap B \right | + \left | A \cap C \right | + \left | C \cap B \right |) + \left | A \cap B \cap C\right |}
\]
 Como podemos observar, la generalización podría llegar a ser más compleja, y no la pondré.
\subsubsection{Principio de Distribución / Dirichlet}
 Si repartimos \(\mathbf{n}\) elementos entre \(\mathbf{m}\geq 1\) células, entonces:
\begin{itemize}
        \item Una célula recibirá hasta \(\left \lceil \frac{n}{m} \right \rceil\) objetos.
        \item Una célula recibirá hasta \(\left \lfloor \frac{n}{m} \right \rfloor\) objetos.
\end{itemize}
 Si consideramos un \(\mathbf{\alpha}\) como una serie de elementos que comprenden un objeto, podemos decir:
\[
        n = \sum_{i=1}\alpha_{i}
\]
 Y por ende:
\begin{itemize}
        \item Podemos decir que \(\forall i\hspace{1mm}\alpha_i < \left \lceil \frac{n}{m} \right \rceil\), y \(\alpha_i \leq \left \lceil \frac{n}{m} \right \rceil -1\) con lo que \(n \leq m \left ( \left \lceil \frac{n}{m}\right \rceil -1\right )\)
\end{itemize}
\subsection{Contando Pares}
 Considerando dos conjuntos \(X\) e \(Y\) que conforman otro conjunto \(S\) llamamos:
\begin{itemize}
        \item \(f_Y(S)\) al número de elementos con la coordenada en las ordenadas.
        \item \(f_X(S)\) al número de elementos con la coordenada en las abcisas.
        \item \(\sum_{y \in Y} f_y(S) =  \sum_{x \in X} f_x(S)\)
\end{itemize}
\subsection{Combinatoria, casos}
\subsubsection{Variaciones}
 Abarca a las distintas formas de agrupar un conjunto, donde el \underline{orden importa}.
\subsubsection{Permutaciones / Biyecciones}
 Es un caso de las variaciones donde el cada grupo abarcamos todos los elementos, donde el \underline{orden importa}.
\subsubsection{Combinaciones}
 Agrupación de elementos donde el orden no importa
\par  En general las podemos agrupar en la siguiente tabla:
\[
        \hspace*{-2.5cm} \textnormal{Abarca todos los Elementos}
        \begin{cases}
                \textnormal{Si e importa el Orden}
                \begin{cases}
                        \textnormal{Si}
                        \begin{cases}
                                \textnormal{Se repiten elementos}
                                \begin{cases}
                                        \textnormal{Si, Permutaciones} = \textnormal{PR}^{t_i}_{m} = \frac{m!}{\prod_{i=1}t_i } \\
                                        \textnormal{No, Permutaciones} = \textnormal{P}_m = m!
                                \end{cases}
                        \end{cases}
                        \\
                        \textnormal{No}\begin{cases}
                                \textnormal{No hay ninguna operación}
                        \end{cases}
                \end{cases} \\
                \textnormal{No e importa el Orden}
                \begin{cases}
                        \textnormal{Si}
                        \begin{cases}
                                \textnormal{Se repiten elementos}
                                \begin{cases}
                                        \textnormal{Si, Variaciones} = \textnormal{VR}_m^n=m^n \\
                                        \textnormal{No, Variaciones} = \textnormal{V}^n_m = \frac{m!}{(m-n)!}
                                \end{cases}
                        \end{cases}
                        \\
                        \textnormal{No}\begin{cases}
                                \textnormal{Se repiten elementos}
                                \begin{cases}
                                        \textnormal{Si, Combinaciones} = \textnormal{CR}_m^n=\binom{m+n-1}{n} \\
                                        \textnormal{No, Combinaciones} = \textnormal{C}^n_m = \binom{m}{n}
                                \end{cases}
                        \end{cases}
                \end{cases}
        \end{cases}
\]
\subsubsection{Emparejamientos, Desarreglos y Circulares}
 Son un caso particular. \\ Los \underline{emparejamientos}, son una ramificación de las \underline{combinaciones}, donde se calcula el numero de formas que se puede repartir un elemento \(m\) en un conjunto de \(m \times n\).
\[
        \boxed{\textnormal{Cm}_m^n= \frac{(mn)!}{(m!)^{n}n!}}
\]
 Los \underline{desarreglos}, son un caso particular de las variaciones donde se calcula el numero de formas que se puede repartir un elemento de formas distintas para que quede en la misma posición:
\[
        \boxed{\textnormal{D}_n = n!\sum_{k=0}^n\frac{(-1)^k}{k!}}
\]
 Las \underline{circulares}, son un caso particular de las \underline{permutaciones}, donde calculamos las formas en las que se puede ordenar un conjunto circular:
\[
        \boxed{\textnormal{PC}_n = (n-1)!}
\]
\subsection{Número Binómico}
 Denominamos al número binómico como \(\binom{n}{k}\), tal que \(n\) y \(k\) son números enteros positivos, tal que equivale a \(\frac{n!}{k!(n-k)!}\).\\
Podemos ver ciertas propiedades que nos servirán a futuro:
\begin{itemize}
        \item \(\binom{0}{0} = \binom{n}{0} = \binom{n}{n} = 1\)
        \item \(\binom{n}{k} = \binom{n}{n-k}\)
        \item Sumar dos binómicos cuya parte baja sea el mismo número más uno es: \(\binom{n}{k} + \binom{n}{k+1} = \binom{n+1}{k+1}\)
        \item La suma de todos los binómicos desde 0 hasta la parte superior equivale a una potencia de dos: \(\sum_{k=0}^n \binom{n}{k} = 2^n\)
\end{itemize}
\subsubsection{Triangulo de Pascal}
\def\N{5}
\tikz[x=0.75cm,y=0.5cm,
        pascal node/.style={font=\footnotesize},
        row node/.style={font=\footnotesize, anchor=west, shift=(180:1)}]
\path
\foreach \n in {0,...,\N} {
                (-\N/2-1, -\n) node  [row node/.try]{Fila \n:}
                \foreach \k in {0,...,\n}{
                                (-\n/2+\k,-\n) node [pascal node/.try] {
                                                $\binom{\n}{\k}$
                                        }}};
\subsubsection{Binomio de Newton}
 Aplicando nuestros conocimientos de los números binómicos, somos capaces de escribir una expresión con la que podemos calcular una expresión del tipo \((X+Y)^n\) con \(X\) e \(Y\) reales y \(n\) un entero positivo:
\[
        \boxed{\sum_{k=0}^n\binom{n}{k}x^{n-k}y^k = \sum_{k=0}^n\binom{n}{k}x^{k}y^{n-k}}
\]