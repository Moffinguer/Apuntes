\subsection{Introducción}
\noindent Dada una función \(f(x)\), podríamos aproximar el valor de una función usando el polinomio de Taylor:
\[
        f(x) = e^x =\sum^n_{k=0} \frac{x^k}{k!}
\]
\noindent Como podemos observar, esta función la podemos represntar como una serie de \(n\) elementos, existen varios tipos, pero nosotros trabajaremos con las más simples:
\begin{itemize}
        \item Series de potencias \(\sum^{\infty}_{n=0} a_n x^n\)
        \item Series geométricas \(\sum^{\infty}_{n=0} x^n\)
\end{itemize}
\subsection{Convergencia}
\noindent Una serie infiníta de términos converge en un valor concreto \(x = c\). Una función puede converger o no. Para calcular la convergencia de una serie utilizamos distintos criterios y a la vez podemos calcular el valor concreto en el que converge:
\[
        \boxed{S_n(x) =A_0 \frac{a_0}{1-a_n}}
\]
\noindent Siendo \(A_0\) el coeficiente del sumatorio, \(a_0\) el primer término de la sucesión y \(a_n\) el término general.
\subsubsection{Criterio del Cociente}
\[
        \boxed{\lim_{n \rightarrow + \infty} \Big |\frac{a_n + 1}{a_n} \Big|
                \begin{cases}
                        \text{Si es Mayor que 1, es convergente} \\
                        \text{Si es Menor que 1, es divergente}  \\
                        \text{Si es Igual a 1, no hay información}
                \end{cases}
        }
\]
\subsubsection{Criterio de la Raiz}
\[
        \boxed{\lim_{n \rightarrow + \infty} \sqrt[n]{\Big |a_n \Big|}
                \begin{cases}
                        \text{Si es Mayor que 1, es divergente}  \\
                        \text{Si es Menor que 1, es convergente} \\
                        \text{Si es Igual a 1, no hay información}
                \end{cases}
        }
\]
\subsubsection{Criterio de Leibniz}
\noindent Solo es aplicable en series alternadas y diremos que la serie es convergente, siempre y cuando:
\begin{itemize}
        \item \(0\leq a_n+1 \leq a_n\) con \(a_n\) decreciente.
        \item  \(\lim_{n\rightarrow +\infty} a_n=0\)
\end{itemize}
\subsection{Intervalo y Radio de Convergencia}
\noindent Aplicaremos el \underline{Teorema de Cauchy-Hadamard} que dice: \par \noindent \textit{Dada una serie del tipo \(\sum^{+\infty}_{n=0}a_n(x-x_0)^n\) diremos que converge en un intervalo \(I_c\) cuando \(I_c \in (x_0 - r, x_0 + r)\) siendo \(r\) el radio de convergencia, que se calcula como \(r > \frac{x}{|x- x_0|}\)}