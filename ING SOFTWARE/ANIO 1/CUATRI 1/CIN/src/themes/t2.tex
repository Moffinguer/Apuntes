\subsection{Introducción}
\begin{tikzpicture}
        % \draw[-,dotted] (2,0) -- (2,3) node[left = 5,above = -55]{\(a\)};
        % \draw[-,dotted] (4,0) -- (4,3) node[left = 5,above = -55]{\(b\)};
        % \draw[-,dotted] (3.2,0) -- (3.2,3) node[above = -30,left = 3]{\(\overline{x}\)};
        \begin{axis}[
                        xtick = \empty, ytick = \empty ,
                        y label style = {at={(0,1)},rotate=-90,anchor=south},
                        axis lines=center,
                        enlargelimits=0.2,
                ]
                \addplot[color=black,smooth,thick,-,domain = 0:10] {cos(deg(x))} node[above = 5]{\(g(x)\)};
        \end{axis}
\end{tikzpicture}
\par \vspace{2mm}
\noindent Un problema común es la aproximación de una función \(g(x)\) tal que pasa por un conjunto de valores \(\Large{\Upsilon} = \left \{ x_0, x_1, ... \right. \left. ,x_n \right \}\). Para lograr obtener \(g(x)\) debemos de obtener una función \(f(x)\) tal que pase por el conjunto de valores establecido.
\subsection{Resolución de forma matricial}
\noindent Considerando un conjunto de puntos de tamaño \(N\) y un conjunto de soluciones de tamaño \(N\) también, podemos construir un sistema de ecuaciones tal que obtenemos un \(P_{N-1}(x)\) tal que el resultado es un polinomio de grado \(N-1\) como máximo. Por ejemplo:
\[
        \Large{\Upsilon} = \left \{ x_0, x_1, \right. \left. x_2 \right \}
\]
\[
        \Large{\Phi}= \left \{ y_0, y_1, \right. \left. y_2 \right \}
\]
\noindent Vemos que el conjunto de valores es de tamaño \(N = 3\), por lo que el grado máximo posible del polinomio es de grado 2:
\[
        P_2(x) = ax^2 + bx + c
\]
\noindent Montamos un sistema de ecuaciones de 3 incógnitas:
\[
        \begin{cases}
                P_2(x_0) = y_0
                \\
                P_2(x_1) = y_1
                \\
                P_2(x_2) = y_2
        \end{cases}
\]
\[
        \begin{pmatrix}
                x^2_0 & x_0 & 1 \\
                x^2_1 & x_1 & 1 \\
                x^2_2 & x_2 & 1
        \end{pmatrix}
        \begin{pmatrix}
                a
                \\
                b
                \\
                c
        \end{pmatrix}
        =
        \begin{pmatrix}
                y_0
                \\
                y_1
                \\
                y_2
        \end{pmatrix}
\]
\noindent Resolvemos la matriz y obtenemos los valores de ``a'', ``b'' y ``c'' y por lo tanto \(P_2(x)\).
\subsection{Resolución por Interpolación de Lagrange}
\noindent Solo podremos usar este método, siempre y cuando la diferencia entre los \underline{soportes}, los valores del conjunto \(\Large{\Upsilon}\) deben de ser regulares, y todos deben de tener la misma diferencia entre sus distancias.
\noindent Para calcular el polinomio a partir de este método, debemos de considerar un conjunto de soluciones \(\Large{\Phi}\) y un conjunto de \underline{Polinomios Auxiliares de Lagrange} \(L_n(x)\), para obtener un \(P_n(x)\):
\[
        \boxed{P_n(x) = \sum^n_{i=0} y_iL_i(x)}
\]
\noindent \hspace{6.2cm} Siendo \(y_i \in \Large{\Phi}\).
\[
        \boxed{L_n(x) = \prod^n_{j=0, \hspace{2mm} i \neq j} \frac{x- x_j}{x_i - x_j}}
\]
\noindent \hspace{6.2cm} Siendo \(x_i \in \Large{\Upsilon}\).
\[
        \boxed{P_n(x) = \sum^n_{i=0} y_i \prod^n_{j=0, \hspace{2mm} i \neq j} \frac{x- x_j}{x_i - x_j}}
\]
\subsection{Método de Newton}
\noindent Este es el método más rapido de los 3 y se calcula a partir de la siguiente expresión:
\[
        \boxed{N_n(x) = \sum^n_{i=0} C_i\prod^{i-1}_{j=0}(x-x_j)}
\]
\noindent De forma que \(P_n(x)\) es:
\[
        \boxed{P_n(x) = \sum^n_{k=0} N_k(x)}
\]
\noindent Lo laborioso de este método es calcular los \(C_i\), que son los soportes que se calculan como proposiciones. Por lo que la expresión de \(P_n(x)\) quedaría algo así:
\[
        \boxed{P_n(x) = f[x_0] + f[x_0,x_1](x-x_0) + ...+ f[x_0,..., x_n](x-x_0)...(x-x_{n-1})}
\]
\noindent Se denominan \underline{proposiciones} y se calculan de la siguiente forma:
\[
        \boxed{f[x_0,...,x_n] = \frac{f[x_1,...,x_n] - f[x_0,...,x_{n-1}]}{x_n - x_0}}
\]
\noindent Siendo el caso base \(f[x_n] = f(x_n)\)
\subsubsection{Cota de Error}
\noindent Se calcula de la siguiente forma:
\[
        \boxed{
        \varepsilon(x) = \frac{\left | f^{n+1}(c) \right |}{(n+1)!}\prod^n_{j=0}\left | x-x_j \right |
        }
\]