\subsection{Introducción}
\noindent Podemos definir dos tipos de ecuaciones:
\begin{itemize}
        \item Ecuaciones Lineales \(\mathbf{f(x) = ax +b}\)
        \item Ecuaciones no lineales \(\mathbf{f(x)}\) tal que tiene varias raices, o soluciones, y se representan como \(\mathbf{\bar{x}}\)
\end{itemize}
\noindent Solamente podemos obtener raices sin recurrir a métodos iterativos, siempre y cuando la ecuación sea de \textbf{grado inferior a 3}, ya que existen expresiones que calculan todas sus raices, y mediante \underline{Ruffini}, solo en casos muy específicos, para cualquier función polinómica.
\par
\vspace{.5cm}
\noindent Definimos la \underline{multiplicidad} de una función con el número de derivadas hasta que la derivada enésima de la función no sea cero con una de las raices la función:
\[
        f(\bar{x}) = 0 \hspace{5mm}f'(\bar{x}) = 0 \hspace{2.5mm} \cdots \hspace{2.5mm} f^n(\bar{x}) \not = 0
\]
Decimos que una función tiene \underline{multiplicidad simple} cuando la primera derivada cumple esta condición.
\subsection{Método de Bisección}
\subsubsection{Cota de Error a priori}
\subsection{Método del Punto Fijo}
\subsubsection{Cota de Error a posteriori}
\subsection{Método de Newton}
\subsubsection{Regla de Fourier}