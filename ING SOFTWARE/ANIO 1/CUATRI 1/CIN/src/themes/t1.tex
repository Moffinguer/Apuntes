\subsection{Introducción}
\noindent Podemos definir dos tipos de ecuaciones:
\begin{itemize}
        \item Ecuaciones Lineales \(\mathbf{f(x) = ax +b}\)
        \item Ecuaciones no lineales \(\mathbf{f(x)}\) tal que tiene varias raices, o soluciones, y se representan como \(\mathbf{\bar{x}}\)
\end{itemize}
\noindent Solamente podemos obtener raices sin recurrir a métodos iterativos, siempre y cuando la ecuación sea de \textbf{grado inferior a 3}, ya que existen expresiones que calculan todas sus raices, y mediante \underline{Ruffini}, solo en casos muy específicos, para cualquier función polinómica.
\par
\vspace{.5cm}
\noindent Definimos la \underline{multiplicidad} de una función con el número de derivadas hasta que la derivada enésima de la función no sea cero con una de las raices la función:
\[
        f(\bar{x}) = 0 \hspace{5mm}f'(\bar{x}) = 0 \hspace{2.5mm} \cdots \hspace{2.5mm} f^n(\bar{x}) \not = 0
\]
Decimos que una función tiene \underline{multiplicidad simple} cuando la primera derivada cumple esta condición.
\subsection{Método de Bisección}
\subsubsection{Cota de Error a priori}
\noindent Aprovecharemos el \underline{Teorema de Bolzano} para poder obtener el número de iteraciones necesarias para alcanzar la precisión deseada:
\[
        \boxed{\varepsilon_n \leq \frac{b-a}{2^n}}
\]
\noindent Siendo \(\mathbf{\varepsilon_n}\) un número que nos indica el número de decimales que queremos obtener, \(\mathbf{b}\) y \(\mathbf{a}\) los extremos del intervalo y \(\mathbf{n}\) el número de iteraciones.\\ Podemos simplificarlo de esta forma, para obtener el número de iteraciones:
\[
        \boxed{n\geq \left \lceil \log_2{\left ( \frac{b-a}{\varepsilon_n} \right )} \right \rceil}
\]
\subsection{Método del Punto Fijo}
\subsubsection{Cota de Error a posteriori}
\noindent La diferencia con el anterior metodo de calcular el error, es que no podemos calcularlo hasta que no obtengamos el valor en la iteración deseada. Se obtiene con la siguiente fórmula:
\[
        \boxed{\varepsilon_n \leq\frac{\left | f(x_n) \right |}{\textnormal{min}_{x \in [a,b]}\left | f'(x) \right |}}
\]
\noindent El único detalle a tener en cuenta es que el denominador debe de ser el valor absoluto de la derivada de \(f(x)\) con \(x\) que pertenezca al intervalo de actuación, y que de el menor valor entre los dos extremos.
\subsection{Método de Newton}
\subsubsection{Regla de Fourier}