\subsection{Observaciones}
\noindent Dado un polinomio de grado \(n\) decimos que lo podemos descomponerlo en factores \((x-x_0)\) siendo \(x_0\) un número real cualquiera:
\[
        f(x) = \sum^k_{j=0} a_jx^{n - j}
\]
\noindent De forma que ahora obtenemos una función tal que:
\[
        p(x) = \sum^k_{j=0} b_j(x-x_0)^{n - j}
\]
\subsection{Condiciones}
\begin{itemize}
        \item \(f(x)\) debe de ser lo suficientemente regular en \(x_0\).
        \item \(f^n(x_0) = T^n_n(x_0)\)
        \item \(T_n(x)\) se expresa en potencias de \((x-x_0)\)
\end{itemize}
\subsection{Polinomio de Taylor}
\[
        \boxed{T_n(x) = \sum^k_{n=0} \frac{f^n(x_0)(x-x_0)^n}{n!}}
\]
\subsection{Teorema de Existencia y Unidad}
\noindent Sea \(f(x)\) tenga dominio sobre todos los reales, y sea \(n\) veces derivable, decimos que existe entonces un \underline{\textbf{único}} \(T_n(x)\) de grado menor o igual que \(n\).
\subsection{Polinomio de Maclaurin}
\noindent Solo cuando \(x_0 = 0\) podemos decir que es un polinomio de Maclaurin
\[
        \boxed{M_n(x) = T_n(x) = \sum^k_{n=0} \frac{f^n(0)(x)^n}{n!}}
\]
\subsection{Resto de Lagrange}
\noindent Considerando que \(c \in <x,x_0>\) podemos obtener el error de la función respecto a la original:
\[
        \boxed{\left | R_n(x) \right | = \Big | \frac{f^{n+1}(c) (x-x_0)^{n+1}}{(n+1)!}\Big |}
\]