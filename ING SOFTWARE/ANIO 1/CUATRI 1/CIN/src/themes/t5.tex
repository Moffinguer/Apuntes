\subsection{Introducción}
 La combinación lineal de funciones \(\sin{x}\) y \(\cos{x}\) genera funciones periódicas y series trigonométricas.
\par  Diremos que una función es periódica cuando \(y = f(x + \kappa) = f(x)\) siendo \(\kappa\) el periodo de la función, o sea, el valor mínimo que verifica la periodicidad.
\subsection{Suma Parcial}
 Esta es la expresión general para las funciones que representan a las series de Fourier:
\[
        \boxed{\sum^{\infty}_{k=0}(a_k \cos{(\kappa w x)} + b_k \sin{(kwx)})}
\]
 El periodo lo podemos definir como \(T = 2\pi\) y \(w = 1\), lo que nos generará una función T-periódica, o en este caso, ``pi-Periódica'', lo que generará una serie Trigonométrica:
\[
        \boxed{S(x) = \frac{a_0}{2}+\sum^{\infty}_{n=1}(a_n \cos{( nw x)} + b_n \sin{(nwx)})}
\]
 Denominaremos a \(a_0\), \(b_n\) y \(a_n\) como los coeficientes de Fourier y se podrán calcular.
\par  \textbf{En caso de que \(f(x)\) sea una función no periódica, la converirémos a ``pi-periódica''}
\subsection{Los coeficientes de Fourier}
\[
        \boxed{a_0 = \frac{2}{T} \int^{T/2}_{-T/2}f(x)\mathrm{d} x}
\]
\[
        \boxed{a_n = \frac{2}{T} \int^{T/2}_{-T/2}f(x)\cos{(xnw)}\mathrm{d} x}
\]
\[
        \boxed{a_0 = \frac{2}{T} \int^{T/2}_{-T/2}f(x)\sin{(xnw)}\mathrm{d} x}
\]
\subsubsection{Paridad}
 Para agilizar los cálculos podemos utilizar la simetría de las funciones:
\[
        \begin{cases}
                \text{\(f(x)\) es par} \Rightarrow f(x) = f(-x)    & S(x) = \frac{a_0}{2} +\sum^{\infty}_{n=1}a_n \cos{(nxw)}
                \\
                \text{\(f(x)\) es impar} \Rightarrow -f(x) = f(-x) & S(x)=\sum^{\infty}_{n=1}b_n \sin{(nxw)}
        \end{cases}
\]
\subsection{Teorema de Dirichlet}
 Mediante este problema comprobamos, \underline{solo en convergencia puntual}, el valor en el que converge la serie en un punto concreto siempre y cuando se cumplan dos condiciones:
\begin{itemize}
        \item \(f(x)\) es continua o tiene un número finito de discontinuidades.
        \item \(f(x)\) tiene un número finito de extremos extrictos.
\end{itemize}
 Con todo esto, concluimos que:
\[
        \begin{cases}
                f(x) \text{\hspace{1cm}Si \(x\) está en un punto continuo}
                \\
                \frac{f(x^-) + f(x^+)}{2} \text{ \hspace{1cm}Si \(x\) se encuentra en una discontinuidad}
        \end{cases}
\]