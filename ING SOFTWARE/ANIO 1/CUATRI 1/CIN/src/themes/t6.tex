\subsection{Introducción}
\subsection{Limite con dos variables}
\subsubsection{Continuidad}
\subsection{Diferenciabilidad}
\subsubsection{Primer Orden}
\subsubsection{Segundo Orden}
\subsubsection{Diferenciales Cruzadas}
\subsection{Diferenciabilidad en un punto}
\subsection{Vector Gradiente}
\subsection{Derivada Direccional}
\subsection{Condición Suficiente de Diferenciabilidad}
\subsection{Condición Necesaria de Diferenciabilidad}
\subsection{Plano Tangente}
\subsection{Extremos Relativos}
\noindent Calculamos las derivadas parciales de primer orden de la función y obtenemos con que valores ambas se anulan:
\[
        \frac{\partial f(x,y)}{\partial x} = 0 \hspace{0.75cm},\hspace{0.75cm} \frac{\partial f(x,y)}{\partial y} = 0
\]
\noindent De aquí obtendremos un conjunto \(\phi = \left \{ \left.  x_0, y_0, ... x_n, y_n\right \} \right.\)\par \noindent Tras esto construiremos una matriz Hessiana, formada por las derivadas parciales de segundo orden, y calcularemos su determinante:
\[\boxed{HF(x,y) = \begin{pmatrix}
                        \frac{\partial f(x,y)}{\partial^2 x}           & \frac{\partial f(x,y)}{\partial x \partial y} \\
                        \frac{\partial f(x,y)}{ \partial y \partial x} & \frac{\partial f(x,y)}{\partial^2 y}
                \end{pmatrix}}
\]
\[
        |HF(x,y)| =  \frac{\partial f(x,y)}{\partial^2 x} \frac{\partial f(x,y)}{\partial^2 y} - \frac{\partial f(x,y)}{\partial x \partial y}\frac{\partial f(x,y)}{ \partial y \partial x} = \Delta
\]
\noindent Tras obtener este determinante, sustituye \(x\) e \(y\) por los valores obtenidos previamente. De esta forma llegaremos a 4 conclusiones:
\[
        \Delta
        \begin{cases}
                \Delta = 0 \text{\hspace{1cm}No hay información suficiente para indicar si es un Punto Crítico}
                \\
                \Delta \neq 0
                \begin{cases}
                        \Delta < 0  \text{\hspace{1cm}Punto de Silla}
                        \\
                        \Delta > 0
                        \begin{cases}
                                \frac{\partial f(x,y)}{\partial^2 x} < 0  \text{\hspace{1cm}Es un máximo relativo}
                                \\
                                \frac{\partial f(x,y)}{\partial^2 x} > 0  \text{\hspace{1cm}Es un mínimo relativo}
                        \end{cases}
                \end{cases}
        \end{cases}
\]