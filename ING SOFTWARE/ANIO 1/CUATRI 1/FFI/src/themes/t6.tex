\subsection{Funciones onda}
 Este tema será el más corto de los 6.
\subsection{Características de las ondas}
 La función de onda se puede representar de la siguiente forma:
\[
        f(x) = f_{\textnormal{o}}\hspace{2mm} \cos{(\kappa x + \rho)} \hspace{2.5mm}\Rightarrow  \hspace{2.5mm}\boxed{f(x,t) = f_{\textnormal{o}}\hspace{2mm}\cos{(\kappa x \pm  wt+\rho)}}
\]
\begin{adjustwidth}{0em}{-8em}
        \begin{tikzpicture}
                \draw[->] (0,0) -- (5,0) node[above = -15] {\(x\)};
                \draw[->] (0,-3) -- (0,4) node[left = 5] {\(f(x)\)};
                \draw [<->] (1,2.2) -- (3,2.2) node[above = 7, left = 20] {\(\lambda\)};
                \draw [<->] (-.5,0) -- (-.5,2) node[left = 10, above = -30] {\(f_{\textnormal{o}}\)};
                \draw[-] (0,0) sin (1,2);
                \draw[-] (1,2) sin (2,-2);
                \draw[-] (2,-2) sin (3,2);
                \draw[-] (3,2) sin (4,-2);
        \end{tikzpicture}
\end{adjustwidth}
 \hspace{-5em} De esta expresión podemos ver ciertas cosas:
\begin{itemize}
        \item \(\mathbf{\lambda}\) es la longitud de onda, la distancia entre dos puntos iguales, se mide en \underline{metros}.
        \item \(\mathbf{\kappa}\) es el número de ondas, se mide en \underline{radiantes entre metros}, se obtiene como \(\boxed{\kappa=\mathbf{\frac{2\pi}{\lambda}}}\)
        \item Existe un \(\mathbf{\pm}\) que indica el desplazamiento en el tiempo de la función, si es \(\mathbf{+}\) indica que se desplaza a la izquierda, sino, a la derecha.
        \item La velocidad \(\boxed{\mathbf{v = \frac{w}{k}}}\) e indica la velocidad a la que se mueve la onda.
        \item Podemos calcular el desfase de una onda como la variación de las fases iniciales, que equivale al número de ondas por la distancia entre ondas \(\boxed{\mathbf{\Delta \rho = \kappa \Delta X}}\). Se mide en radianes.
        \item La velocidad angular \(\mathbf{w}\) y la frecuencia también las podemos calcular \(\boxed{\mathbf{f = \frac{w}{2\pi}}}\) y se mide la frecuencia en Hercios \textbf{Hz} y la velocidad angular en \textbf{m/sec}.
        \item El periodo indica el tiempo que tiene que pasar para que la función vuelva a pasar por ese punto, se calcula como la inversa de la frecuencia \(\boxed{\mathbf{T = \frac{1}{f}}}\) y se mide en \textbf{segundos}.
\end{itemize}
\subsection{Ondas electromagnéticas}
 Este es un caso particular de las ondas, ya que se forman por la intersección entre una onda magnética y una eléctrica.
\[
        \boxed{\vec{B}(x,t) = B_{\textnormal{o}} \cos{(\kappa x \pm  wt+\rho)} \hat{B}}
\]\[
        \boxed{\vec{E}(y,t) = E_{\textnormal{o}} \cos{(\kappa y \pm  wt+\rho)} \hat{E}}
\]
 Sabemos además que los vectores normales de ambas ondas son perpendiculares entre si, por lo que:
\[
        \mathbf{\boxed{\hat{E} \times \hat{B} = \hat{n_{\textnormal{dir}}}}}
\]
 Como datos a tener en cuenta para saber su orientación y calcular sus variables:
\begin{itemize}
        \item Estas ondas se mueven a la velocidad de la luz \(\mathbf{\boxed{c = \frac{1}{\sqrt{\epsilon_{\textnormal{o}}\mu_{\textnormal{o}}}}}\approx 3.10^8\hspace{2mm}\textnormal{m/sec}}\) Esto también es igual a \(\mathbf{\boxed{c=\frac{E_{\textnormal{o}}}{B_{\textnormal{o}}}} = \frac{w}{\kappa}}\).
        \item Si el signo de \(\boxed{\mathbf{ \kappa x } = \mathbf{wt}}\) entonces la onda electromagnética se mueve en sentido del eje negativo, sino, en el positivo.
\end{itemize}
\subsubsection{Vector de Poynting}
\[
        \boxed{\vec{S} = \frac{\vec{E}\times \vec{B}}{\mu_0}}
\]
 Siendo \(\vec{E}\) y \(\vec{B}\) los vectores con amplitudes \textbf{no eficaces}.
\subsubsection{Intensidad}
 Considerando el \underline{vector de Poynting} podemos calcular la intensidad como la amplitud de este vector entre 2.
\[
        \boxed{I = \frac{E_0 B_0}{2\mu_0} = \frac{P}{S}}
\]
 Siendo \textbf{P} y \textbf{S} la potencia de la onda y la superficie a la que afecta.
\par  Por separado podemos calcular la energía de cada onda:
\[
        \boxed{\mu_E = \frac{E^2\epsilon_0}{2}}
\]
\[
        \boxed{\mu_B = \frac{B^2}{\epsilon_0 2}}
\]