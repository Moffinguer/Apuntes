\subsection{Fuerza Magnética}
\begin{tikzpicture}
        \draw [->] (0,0) -- (0,2) node[left = 1]{\(\vec{F_m}\)};
        \draw [->](0,0) -- (2,0) node[above = 1]{\(\vec{B}\)};
        \draw [->] (0,0,0) -- (1.7,0,1.5) node[above = -20]{\(\vec{v}\)};
\end{tikzpicture}
\newline
\noindent La fuerza magnética actua sobre las cargas en movimiento y se define, para una carga en movimiento como:
\[
        \vec{F_m} = q\hspace{2mm} \vec{v} \times \vec{B}
\]
\noindent \(\bm{\vec{F_m}}\) es perpendicular \(\bm{\perp}\) a \(\bm{\vec{B}}\) y a \(\bm{\vec{v}}\) y siempre irá en el sentido que indique la carga, la velocidad y el campo, por lo que la Fuerza magnética es proporcional a la carga, el campo y a la velocidad.
\\
Algunas propiedades que nos vendrán bien a la hora de calcular son las siguientes, respecto al producto vectorial:
\begin{enumerate}
        \item \(\bm{\hat{\imath} \times \hat{\imath} = 0}\) Y lo mismo ocurre para cualquier otra combinación.
        \item \(\bm{\hat{\imath}\times \hat{\jmath} = \hat{k}}\), \(\bm{\hat{\jmath}\times\hat{k}=\hat{\imath}}\),\(\bm{\hat{\imath}\times\hat{k}=\hat{\jmath}}\).
        \item \(\bm{\vec{v} \times \vec{B}} = - \vec{B} \times \vec{v}\)
        \item El modulo es \(\bm{\left | \vec{v} \right |\left | \vec{B} \right |\sin{\alpha}}\).
        \item La dirección, es la perpendicular entre \(\vec{v}\) y\(\vec{B}\).
        \item El sentido viene indicado por la regla de la mano derecha.
\end{enumerate}
Además consideraremos \(\bm{\bigotimes}\) para cuando el campo vaya hacia dentro del papel y \(\bm{\bigodot}\) cuando sea hacia fuera.
\subsection{Ley de Lorentz}
\subsubsection{Selector de Velocidades}
\subsection{Fuerza Magnética en un hilo conductor}
\subsection{Ley de Biot y Savart}
\subsubsection{Espira Circular}
\subsection{Ley de Ampere}
\subsubsection{Solenoide }