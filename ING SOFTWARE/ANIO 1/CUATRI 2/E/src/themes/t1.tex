\subsection{Introducción}
\noindent Cualquier sentencia la podemos descomponer en un lenguaje formal del cual podemos analizar su veracidad y su sentido, sin error a no comprenderlo. Para lograr esto debemos seleccionar las ideas y dividirlas en proposiciones, para así poder juntarlas.
\subsection{Sintaxis básica}
\noindent Usaremos 5 operadores para replicar el sentido de las sentencias en proposiciones:
\begin{itemize}
        \item \(\mathbf{\neg}\): Operador de negación.
        \item \(\mathbf{\land}\): Operador de conjunción.
        \item \(\mathbf{\lor}\): Operador de disyunción.
        \item \(\mathbf{\rightarrow}\): Operador ``implica que...''.
        \item \(\mathbf{\leftrightarrow}\): Operador ``si y solo si''.
\end{itemize}
\noindent Podemos eliminar los paréntesis de una expresión, pero para mantener el sentido original, debemos de respetar una regla, además de no quitar los parentesis que violen parte del sentido original, y es colocar o poner los paréntesis en función del orden establecido arriba en la lista.
\subsubsection{Fórmulas Proposicionales}
\[
        (\neg p \rightarrow (\neg(q \land t) \lor p) ) \leftrightarrow \neg q
\]
\noindent Esta expresión es una fórmula proposicional, hemos añadido los paréntesis para mejorar su legibilidad.
\subsection{Árboles de análisis y Subfórmulas}
\subsubsection{Árboles de análisis}
\noindent Dada la expresión de arriba podemos dividirla en segmentos para despedazarla en subfórmulas:
\[\neg p \rightarrow (\neg(q \land t) \lor p) \hspace{2cm} \neg q\]
\[\hspace{.1cm}\neg p \hspace{.5cm} \neg(q \land t) \lor p\hspace{2cm} q\]
\[\hspace{-1cm}p \hspace{.5cm} \neg(q \land t) \hspace{.2cm} p\]
\[\hspace{-.5cm}q \land t\]
\[\hspace{-.5cm}q \hspace{.5cm} t\]
\noindent Hemos dividido la expresión en partes.
\subsubsection{Subfórmulas}
\noindent Una subfórmula no es más que una proposición compleja, o no, que forma parte de la fórmula. Por ejemplo, \(q\) es una expresión atómica y \(\neg(q \land t\) es una subfórmula compleja.
\subsection{Tablas de Verdad}
\vspace{.75cm}
\begin{table}[h]
        \begin{tabular}{l|l|l|l|l|l|l|l|l}
                \(I_n(T)\) & \(F\) & \(G\) & \(\neg F\) & \(\neg G\) & \(F\land G\) & \(F\lor G\) & \(F\rightarrow G\) & \(F\leftrightarrow G\) \\ \hline
                \(I_0\)    & 1     & 1     & 0          & 0          & 1            & 1           & 1                  & 1                      \\ \hline
                \(I_1\)    & 0     & 0     & 1          & 1          & 0            & 0           & 1                  & 1                      \\ \hline
                \(I_2\)    & 1     & 0     & 0          & 1          & 0            & 1           & 0                  & 0                      \\ \hline
                \(I_3\)    & 0     & 1     & 1          & 0          & 0            & 1           & 1                  & 0
        \end{tabular}
\end{table}
\noindent Mediante esta tabla seremos capaces de extraer todas las conclusiones que querramos. Siendo \(G\) y \(F\) sentencias complejas lógicas.
\subsection{Interpretaciones}
\noindent Como su nombre indica no son más que las combinaciones que tiene una sentencia al variar los valores de sus variables, a cada interpretación verdadera la denominamos como ``\textit{modelo}''. Los modelos se expresan de esta forma:
\[
        \boxed{I \models F}
\]
\noindent Las podemos dividir en distintos tipos:
\begin{itemize}
        \item \textbf{Satisfacibles}: Fórmulas que tienen algún modelo que verifica que es cierta la proposición. Se le oponen las interpretaciones \textbf{Insatisfacibles}).
        \item \textbf{Tautologias}: Todas las interpretaciones son un modelo, son ciertas. Se le denominan \textit{válidas}. Se escriben como:
              \[ \models F\]
        \item Se le oponen las \textbf{Contigentes}, que verifican que no es una tautología pero tampoco todas son instatisfacibles.
\end{itemize}
\subsection{Conjuntos}
\noindent Varias fórmulas que podemos agrupar, dan lugar a un conjunto:
\[
        \boxed{\mathrm{S} = \left \{ \left. p_1, p_2, ..., p_n \right \} \right. \hspace{.5cm} \leftrightarrow \hspace{.5cm} I \models \left \{ \left. \mathrm{S} \right \} \right.}
\]
\noindent Al igual que las fórmulas, podemos distinguir distintos tipos de conjuntos:
\begin{itemize}
        \item \textbf{Consistente}: Se le llama a aquel conjunto que posee algún modelo, se le opone los \textit{inconsistentes}
        \item \textbf{Consecuencia Lógica}: Todos los modelos de S son modelos de la fórmula o cuando hay inconsistencias en el conjunto. Lo representamos como:
              \[
                      \boxed{\mathrm{S} \models F}
              \]
              \noindent Siendo \(F\) la fórmula.
\end{itemize}