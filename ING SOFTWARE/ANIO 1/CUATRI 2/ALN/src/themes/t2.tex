\subsection{Introducción}
Dado un sistema \(Ax=b\), si \(A\) es invertible, es decir, el sistema es \underline{compatible determinado}, por lo que podemos calcular la solución como \(\hat{x} = A^{-1}b\), siendo \(\hat{x}\) la única solución posible. \par
Usamos este método cuando el sistema es pequeño, pero si la matriz \(A\) es extremadamente grande, podemos usar métodos de aproximación que nos ayuden a acercarnos a la solución, con un error muy cercano a 0. \par
Para lograr esto usaremos el \underline{\textbf{radio espectral}} y el \underline{\textbf{autovalor}}.
\subsection{Radio Espectral}
Consideramos \(Q\) como una matriz cuadrada, por lo que un \textbf{autovalor} es un escalar \(\lambda\) para el que existe un vector \(x\) no nulo tal que:
\[
        Qx = \lambda x
\]
Tras esto podemos calcular los autovalores como:
\[
        \boxed{\left | Q - \lambda I \right | = 0}
\]
De esta forma solo tendremos que calcular un determinante para esta matriz resultante tal que sus raices son sus autovalores:
\[
        \boxed{\text{pQ}(\lambda) = \left | Q - \lambda I \right |}
\]
Así obtendremos el radio espectral, que será el autovalor de mayor valor, absoluto:
\[
        \boxed{\rho(\lambda) = \left \{ \left | \lambda \right | : \lambda \text{\hspace{2mm} autovalor de \hspace{2mm}} Q \right \}}
\]
\subsection{Descomposición}
¿Cómo calculamos la matriz \(Q\)?, simple, cualquier matriz cuadrada se puede descomponer en la suma de dos matrices (una de ellas es invertible):
\[
        A = M + N \hspace{2em} Ax = b
\]
\[
        \left (  M + N\right )x = b
\]
\[
        Mx = b - Nx
\]
\[
        x = M^{-1}b + M^{-1}Nx = C +Qx
\]
De esta forma, de forma general podemos obtener la solución en la iteración enésima:
\[
        \boxed{\hat{x}_{n} = Qx_{n-1} + C}
\]
Podemos calcular el error con \(Q^n\), cuanto más cercana a cero sea esa matriz, entonces más precisa es la solución. \par Es decir, calculamos el radio espectral de:
\[
        \boxed{\rho(-M^{-1}N) = 0}
\]
Ahora, considerando que las matrices \(M\) y \(N\), las podemos seguir descomponiendo, podemos descomponer \(A\) en 3 matrices (triangular superior e inferior y la diagonal)
\[
        A = D + U + L
\]
\subsubsection{Método de Jacobi}
\[
        \boxed{\hat{x_n} = Jx_{n-1} + C} \hspace{3em} J = -D^{-1}(L+U) \hspace{2em} C = D^{-1}b
\]
\subsubsection{Método de Gauss-Seidel}
\[
        \boxed{\hat{x_n} = \text{GS}x_{n-1} + C} \hspace{3em} \text{GS} = -(D+L)^{-1}U \hspace{2em} C = (D+L)^{-1}b
\]