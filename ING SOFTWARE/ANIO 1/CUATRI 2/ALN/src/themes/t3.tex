\subsection{Introducción}
En la realidad, los valores de nuestro sistema no son conocidos o exactos, lo que afecta a nuestro sistema, para esto condicionaremos el sistema (para obtener un error muy bajo)
\par Cuando peor condicionado esté el sistema, más grande será el error.
\subsection{Espacio Vectorial}
Sobre un cuerpo \(\mathbb{K}\) (\(\mathbb{R} \) o \(\mathbb{C} \)), es un cuerpo \(\mathcal{V}\) que tiene dos operaciones (suma y producto). Siendo \((u,v \in \mathcal{V})\), podemos usar estas propiedades:
\begin{itemize}
        \item Suma:
              \begin{itemize}
                      \item Propiedad Conmutativa
                      \item Propiedad Asociativa
              \end{itemize}
        \item Producto:
              \begin{itemize}
                      \item \((\alpha, \beta)(v+u) = \alpha v + \alpha u + \beta v + \beta u\)
                      \item  \((\alpha\beta)u = u(\alpha\beta)\)
              \end{itemize}
\end{itemize}
\subsubsection{Ejemplos}
Por parte de los vectores, se pueden escribir así:
\[
        \mathbb{R}^n = \left \{ n\left ( u_1,..., u_n \right ) : u_1,...,u_n \in \mathbb{R} \right \}
\]
Por parte de las matrices:
\[
        \mathcal{M}_n(\mathbb{R}) = \left \{ A_{\left ( aij \right )} \text{\hspace{2mm}matrices cuadradas de orden \(n\) con \hspace{2mm}} aij \in \mathbb{R} \right \}
\]
\subsection{Normas Vectoriales y Matriciales}
\subsubsection{Vectores}
Es un \textbf{espacio vectorial} \(\mathcal{V}\) sobre \(\mathbb{R}\) tal que \(\left \| \cdot  \right \|: \mathcal{V} \rightarrow\left [ 0, \infty \right )\) cumple:
\begin{itemize}
        \item \(\left \| u \right \| = 0 \Leftrightarrow  u = 0\)
        \item Propiedad homogénea: \(\left \| \lambda u \right \| = \left | \lambda \right |\left \| u \right \|\)
        \item \(\left \| u + v \right \| \leq  \left \| u \right \| + \left \| v \right \|\)
\end{itemize}
De esta forma podemos calcular también la distancia entre dos vectores:
\[
        d(u,v) = \left \| u - v \right \|
\]
Es un \textbf{espacio normado}, un espacio vectorial \(\mathcal{V}\) dotado de una forma:
\begin{itemize}
        \item \(\left \| u \right \|_{\infty} = \max \left \{ \left | u_1 \right |, ..., \left | u_n \right | \right \} \)
        \item \(\left \| u \right \|_k=\sqrt[k]{\sum_{n=1} u_n^k}\)
        \item Norma euclídea: \(\left \| u \right \|_2=\sqrt[2]{\sum_{n=1} u_n^2}\)
\end{itemize}
\subsubsection{Matrices}
Una \textbf{norma matricial}, se define en un espacio de matrices, normas sobre \(\mathcal{M}_n(\mathbb{R})\) que cumplen:
\begin{itemize}
        \item \(\left \| AB \right \| \leq  \left \|  A\right \|\left \|  B\right \|\)
        \item Norma matricial con la vectorial: \(\left \| Au \right \| \leq \left \| A \right \| \left \| u \right \|\)
\end{itemize}
Ejemplos de normas son:
\begin{itemize}
        \item \(\left \| A \right \| = \max _{u \neq 0} \frac{\left \| Au \right \|}{\left \| u \right \|}\)
        \item \underline{La máxima suma de las \textbf{columnas}} \(\left \| A \right \|_\infty = \max  \sum_{j=1} \left | a_{ij} \right |\)
        \item \underline{La máxima suma de las \textbf{filas}} \(\left \| A \right \|_1 = \max  \sum_{i=1} \left | a_{ij} \right |\)
        \item \textbf{\underline{Norma espectral}}: \(\left \| A \right \|_2 = \sqrt{\rho(A^tA)}\)
        \item \textbf{\underline{Norma de Frobenius}}, es la suma de todos los elementos de la matriz, al cuadrado: \(\sqrt{\sum_{i=1}\sum_{j=1} a_{ij}^2}\).\par O la suma de los elementos de la matriz traspuesta por si misma: \(\sqrt{\delta(A^t,A)}\) \par Es similar a la norma espectral.
\end{itemize}
\subsection{Número de Condición de una Matriz}
Dado un sistema \(Ax=b\), con \(A\) invertible y \(b \neq 0\), \(b\) se modificará por \(b_p\):
\[
        \spalignsys{
                Ax = b \hspace{2em} x_0 = A^{-1}b;
                Ax_p = b_p \hspace{2em} x_p = A^{-1}b_p;
        }
\]
La solución de \(x_p\) tendrá un error, \(\mathbf{C} = \left \| b - b_p \right \|\), que llamaremos \textbf{error absoluto}. \par Si:
\[\varepsilon \simeq 0 \Rightarrow  \left \| x_0 - x_p \right \| = \left \| A^{-1} \left ( b -b_p \right )\right \| \leq \left \| A^{-1} \right \| \left \| b - b_p \right \|\]
\underline{Esto significa que nos interesa que \(\left \| A^{-1} \right \|\) sea lo más pequeño posible, ya que \(\left \| b - b_p \right \|\) será muy pequeño}
\par \vspace{1em}
Por otro lado:
\[
        \left \| b \right \| \leq \left \| A x_0 \right \| \hspace{1em} \Rightarrow \hspace{1em} \left \| x_0 \right \| \leq \frac{\left \| b \right \|}{\left \| A \right \|}
\]
Entonces, el error relativo cumple lo siguiente:
\[
        \boxed{\frac{\left \| x_0 - x_p \right \|}{\left \| x_0 \right \|}\leq \left \| A \right \|\left \| A ^{-1} \right \| \frac{\left \| b - b_p \right \|}{\left \| b \right \|} < \lambda} \hspace{2em} \boxed{\text{cond}\left ( A \right )\varepsilon < \lambda}
\]
Considerando \(\lambda\) el error total, y \(\varepsilon\) el error de precisión, solo tendremos que tomar la precisión del error de precisión como:
\[
        \varepsilon > \frac{\left \|  b - b_p\right \|}{\left \| b \right \|}
\]
Podemos concluir que lo que indica si el error es grande o no es el condicionamiento de A, cuanto menor sea este valor, menor el error.
\par Si \(\varepsilon\) no sobrepasa el valor \(\lambda\), diremos que el sistema está bien condicionado.[Ejemplo: \(\lambda = 10^{-3} \rightarrow \varepsilon = 10^{-4}\)]
\par En definitiva:
\[
        \text{cond}(A)\frac{\mathbf{C}}{\left \| b_0 \right \|} < \lambda \hspace{1em}  ; \hspace{1em} \varepsilon > \frac{\mathbf{C}}{\left \| b_0 \right \|}
\]
\subsection{Transformaciones y Condicionamiento}
Sigue siendo laborioso tantos cálculos, por ende usaremos estas propiedades para simplificar las tareas:
\begin{itemize}
        \item \(\text{cond}(A) \geq 1\) pero cuanto más cerca a 1 esté, mejor condicionado.
        \item si tenemos dos sistemas  \[
                      \spalignsys{
                              Ax = b ;
                              Bx = c;
                      }
              \] El que tenga un condicionamiento más cercano a 1, será mejor.
        \item Llamamos \(\lambda_{\text{min}}\) y \(\lambda_{\max }\) a los autovalores de \(A^tA\) de mayor y menor módulo, por lo que tenemos:
              \[
                      \boxed{\text{cond}_2(A) = \left \| A \right \|_2 \left \| A^{-1} \right \|_2=\sqrt{\frac{\lambda_{\max }}{\lambda_{\text{min}}}}}
              \]
        \item Llamamos a \(\lambda_n\) a los autovalores de \(A^tA\), por lo que tenemos:
              \[
                      \boxed{\text{cond}_\text{F}(A) = \left \| A \right \|_\text{F} \left \| A^{-1} \right \|_\text{F}=\sqrt{\sum_{i=1}\lambda_i \sum_{i=1}\frac{1}{\lambda_i}}}
              \]
\end{itemize}
Para lograr una transformación eficiente, debemos de encontrar una matriz \(U\) tal que cumpla dos propiedades:
\begin{itemize}
        \item \(\text{cond}(UA)\leq \text{cond}(A)\)
        \item La matriz \(U\) debe de ser unitaria, ortogonal:
              \begin{itemize}
                      \item \(U^{-1} = U^t\)
                      \item \(U^tU = I\)
                      \item \(\text{cond}(UA)_2 = \text{cond}(A)_2\)
                      \item \(\text{cond}(UA)_\text{F} = \text{cond}(A)_\text{F}\)
              \end{itemize}
\end{itemize}
\subsection{Transformaciones Householder}
Dado un vector \(v \in \mathbb{R}^n\), se define la \underline{transformación Householder asociada a \(v\)}:
\[
        \boxed{H_v = I - \frac{2}{v^tv}vv^t}
\]
De forma que:
\[
        v = \begin{pmatrix}
                v_1    \\
                \vdots \\
                v_n
        \end{pmatrix}
\]
Así, podemos calcular los dos productos de una forma bastante eficiente:
\[
        \boxed{vv^t = \begin{pmatrix}
                        v^2_1   & v_1v_2 & \cdots & v_1 v_n \\
                        v_2v_1  & v_2^2  & \cdots & v_2 v_n \\
                        \vdots  & \vdots & \ddots & \vdots  \\
                        v_n v_1 & v_nv_2 & \cdots & v_n^2
                \end{pmatrix}}
        \hspace{3em}
        \boxed{v^tv = \left \| v \right \|^2_2 = \sum_{i=1}v_i^2}
\]
\subsection{Método QR}
Para lograr obtener \(v\) debemos de obtener dos vectores \(u\) y \(w\) que tendrán que tener la misma norma euclídea, y repetiremos el proceso tantas veces como filas tenga \(A\)
\[
        (A|b) = \begin{pmatrix}
                1  & -3 & -2 & 3  & |2  \\
                1  & 0  & 3  & 2  & | 2 \\
                1  & -4 & 4  & 1  & | 2 \\
                -1 & 3  & 1  & -4 & | 0 \\
        \end{pmatrix}
\]
El vector \(u_1\) se corresponderá con la columna 1 de la matriz \(A\) y \(w_1\) con un vector del mismo tamaño que \(u_1\) con todos los valores a 0, excepto su posición 1.
\par En caso de trabajar con la iteración enésima, \(u_n\) será el vector enésimo de la anterior transformación y \(w_n\) un vector con todos los valores a cero, salvo el de la posición enésima, todos los valores anteriores a esta posición tendrán el mismo valor que el del vector \(u_n\).
\[
        u_1 = \begin{pmatrix}
                1 \\
                1 \\
                1 \\
                -1
        \end{pmatrix}
        w_1 = \begin{pmatrix}
                \left \| u_1 \right \|^2 \\
                0                        \\
                0                        \\
                0
        \end{pmatrix}
        \left \| u_1 \right \|^2 = 2
\]
\[
        v_1 = u_1 - w_1 = \begin{pmatrix}
                -1 \\
                1  \\
                1  \\
                -1
        \end{pmatrix}
\]
\[
        H_{v_1} = \begin{pmatrix}
                1 & 1 & 1 & 1 \\
                1 & 1 & 1 & 1 \\
                1 & 1 & 1 & 1 \\
                1 & 1 & 1 & 1
        \end{pmatrix}
        - \frac{2}{4}
        \begin{pmatrix}
                1  & -1 & -1 & 1  \\
                -1 & 1  & 1  & -1 \\
                -1 & 1  & 1  & -1 \\
                1  & -1 & -1 & 1
        \end{pmatrix}
        H_{v_1}(A|b) = \begin{pmatrix}
                2 & -5 & 2  & 5  & | 1  \\
                0 & 2  & -1 & 0  & | -1 \\
                0 & -2 & 0  & -1 & | -1 \\
                0 & 1  & 5  & -2 & |3
        \end{pmatrix}
\]
Ahora que tenemos la primera iteración hacemos las siguientes:
\[
        u_2 = \begin{pmatrix}
                -5 \\
                2  \\
                -2 \\
                1
        \end{pmatrix}
        w_2 = \begin{pmatrix}
                -5                       \\
                \left \| u_2 \right \|^2 \\
                0                        \\
                0
        \end{pmatrix}
        \left \| u_2 \right \|^2 = 3
\]
\[
        v_2 = u_2 - w_2 =\begin{pmatrix}
                0  \\
                -1 \\
                -2 \\
                1
        \end{pmatrix}
\]
\[
        H_{v_2}(A|b) = \begin{pmatrix}
                2 & -5 & 2 & 5  & |1  \\
                0 & 3  & 1 & 0  & | 1 \\
                0 & 0  & 4 & -1 & |3
                \\ 0 &  0& 3 &  -2& |1
        \end{pmatrix}
\]
Ahora la ultima:
\[
        u_3 = \begin{pmatrix}
                2 \\
                1 \\
                4 \\
                3
        \end{pmatrix}
        w_3 = \begin{pmatrix}
                2                        \\
                1                        \\
                \left \| u_3 \right \|^2 \\
                0
        \end{pmatrix}
        \left \| u_3 \right \|^2 = 5
\]
\[v_3 = u_3 - w_3 = \begin{pmatrix}
                0  \\
                0  \\
                -1 \\
                3
        \end{pmatrix}\]
\[
        H_{v_3}(A|b) = \begin{pmatrix}
                2 & -5 & 2 & 5  & |1 \\
                0 & 3  & 1 & 0  & |1 \\
                0 & 0  & 5 & -2 & |3 \\
                0 & 0  & 0 & 1  & |1
        \end{pmatrix}
\]
Ya solo tendremos que resolver el sistema resultante usando Gauss