\subsection{Conceptos Básicos}
\subsubsection{Base de un Espacio Vectorial}
Sea \(\mathcal{V}\) un espacio vectorial sobre \(\mathbb{R} \left(\mathcal{V} = \mathbb{R}^n\right) \) y \(\left(v_1,\cdots,v_n\right) \in \mathcal{V} \), denotaremos:
\begin{enumerate}
        \item  Una \underline{combinación lineal} (CL) de \(\left(v_1,\cdots,v_n\right)\) a cualquier expresión tal que:
              \[
                      \sum_{i=1}\alpha_i v_i \hspace{2em} \text{Con } \alpha_i \in \mathbb{R}
              \]
        \item  \(\left(v_1,\cdots,v_n\right)\) son \underline{linealmente independientes} (LI) si:
              \[
                      \sum_{i=1}\alpha_i v_i = 0
              \]
        \item  \(\left(v_1,\cdots,v_n\right)\) son \underline{linealmente dependientes} (LD) si no son \textbf{CL} del resto, es decir si \(\exists u,v \in \mathcal{V} \hspace{.25em} v = \alpha u\), es decir si ambos vectores son proporcionales entre si.
              \(\left\{v_1,\cdots,v_n\right\} \) es un \underline{sistema generador} (Sg), de \(\mathcal{V}\) si cualquier vector de \(\mathcal{V}\) se puede expresar como \textbf{CL} de \(\left\{v_1,\cdots,v_n\right\} \)
        \item  Llamamos \underline{base} de un espacio vectorial \(\mathcal{V}\) a cualquier conjunto de vectores \(\mathcal{B} = \left\{v_1,\cdots, v_n\right\} \) que sean \textbf{Sg} y \textbf{LI} (Como dato a tener en cuenta, mientras no nos digan lo contrario, trabajaremos sobre la \textbf{base canónica} \(\mathbf{C} = \left\{e_1,e_2,\cdots, e_n\right\} \)) que tendrán una estructura como esta:
              \[
                      e_1 =\begin{pmatrix}
                              1      \\
                              0      \\
                              \vdots \\
                              \vdots \\
                      \end{pmatrix}
                      e_2 =
                      \begin{pmatrix}
                              0      \\
                              1      \\
                              \vdots \\
                              \vdots
                      \end{pmatrix}
                      \cdots
                      e_n =
                      \begin{pmatrix}
                              0      \\
                              0      \\
                              \vdots \\
                              1
                      \end{pmatrix}
              \]
        \item Todas las bases de \(\mathcal{V}\) tienen el mismo número de vectores, se denomina \underline{dimensión} \(\text{dim}\left(\mathcal{V}\right) \)
              \par Por esto, \(\forall u \in \mathcal{V}\), existe una forma de expresarlo como \textbf{Cl} de los vectores de la base en la que se encuentra:
              \[
                      \mathcal{B} = \left\{v_1,\cdots, v_n\right\}  \Rightarrow u = \sum^k_{n=1}\alpha_n v_n =\sum^k_{n=1}\beta_n v_n
              \]
        \item A los coeficientes, \(\left(\alpha, \beta\right) \), se les llama \underline{coordenadas de \(u\)} respecto de \(\mathcal{B}\), se escriben los vectores entonces así:
              \[
                      \boxed{u = v_1\left(\alpha_1,\cdots,\alpha_n\right),\cdots,v_n\left(\beta_1,\cdots,\beta_n\right)_\mathcal{B} }
              \]
              Por ejemplo:
              \[
                      \mathcal{B} = \left\{v_1\left(2,5\right) , v_2\left(1,-1\right) \right\} \hspace{2em} u=\left(4,3\right)
              \]
              \[
                      \begin{pmatrix}
                              4
                              \\
                              3
                      \end{pmatrix}
                      =\alpha_1
                      \begin{pmatrix}
                              2
                              \\
                              5
                      \end{pmatrix}
                      +\alpha_2
                      \begin{pmatrix}
                              1
                              \\
                              -1
                      \end{pmatrix}
                      \Rightarrow
                      \spalignsys{
                              u = (1\text{,}2)_\mathcal{B};
                              u = (4\text{,}3)_\mathbf{C};
                      }
              \]
\end{enumerate}
Para terminar vamos a hablar de las bases.
\par Partiendo de la definición de espacio vectorial, dada anteriormente y \(w,u \in \mathcal{V}\) de los que conocemos sus coordenadas en base \(\mathcal{B}\) vemos:
\[
        w_\mathcal{B} = \sum^k_{i=1}\alpha_i u_{i\mathcal{B}} \Leftrightarrow \begin{pmatrix}
                u_{1\mathcal{B}} & \cdots & u_{k\mathcal{B}}
        \end{pmatrix}
        \begin{pmatrix}
                \alpha_1 \\
                \vdots   \\
                \alpha_k
        \end{pmatrix}
        =w_\mathcal{B}
\]
De esta forma \(w\) es \textbf{CL} de \(u\) solamente si el sistema de ecuaciones con el que trabajemos,\(Ax=b\), es compatible.
\begin{itemize}
        \item[\(\Rightarrow\)] \(\left(u_1,\cdots,u_n\right) \) son \textbf{LI} si y solo si, el rango de nuestra matriz \(A\) es igual al tamaño de este vector \(u\): \(\text{Rango}\left(A\right) = n \) y \(\left\lvert A\right\rvert \neq 0\)
        \item[\(\Rightarrow\)] \(\left\{u_1,\cdots,u_n\right\} \) es base de \(\mathcal{V}\), por lo que \(\text{dim}\left(\mathcal{V}\right) = n \) y el vector \(u\) tiene todas sus componentes \textbf{LI}, si se aplica cualquier transformación elemental por filas.
\end{itemize}
\subsubsection{Cambio de Base}
Supongamos que tenemos dos bases \(\mathcal{B}\)\ y \(\mathcal{D}\), de un espacio vectorial \(\mathcal{V}\) y un vector genérico \(v\), digamos que para expresar el vector \(v\) en base \(\mathcal{B}\) a la base \(\mathcal{D}\):
\[
        \boxed{v_\mathcal{D} =
                \begin{pmatrix}
                        u_{1\mathcal{D}} & |\cdots| & u_{n\mathcal{D}}
                \end{pmatrix}
                \begin{pmatrix}
                        v_{1\mathcal{B}}
                        \\
                        \vdots
                        \\
                        v_{n\mathcal{B}}
                \end{pmatrix}}
        \hspace{3em}
        \boxed{v_\mathcal{B} =
        \begin{pmatrix}
                u_{1\mathcal{D}} & |\cdots | & u_{n\mathcal{D}}
        \end{pmatrix}^{-1}
        \begin{pmatrix}
                v_{1\mathcal{D}}
                \\
                \vdots
                \\
                v_{n\mathcal{D}}
        \end{pmatrix}}
\]
Solo podremos hacer esto cuando las dos bases con las que trabajemos, se encuentren en el mismo espacio vectorial.
\par Ejemplo:
\[
        \mathcal{B} = \left\{v_1, v_2, v_3, v_4\right\}  \hspace{1em} \mathcal{D} = \left\{u_1, u_2, u_3, u_4\right\} \in \mathbf{R}^4 \Rightarrow
        \spalignsys{
                u_1 = 3v_1 - 2_v3 + v_4;
                u_2 = -v_1 + v_2 - v_4;
                u_3 = v_1 - 2v_2 + v_3 + 2v_4;
                u_4 = v_2 - v_3
        }
\]
Podemos sacar la siguiente ecuacion de la base \(\mathcal{D}\) respecto de \(\mathcal{B}\):
\[
        \mathcal{D} = \left\{\left(3,0,-2,1\right)_\mathcal{B},
        \left(-1,1,0,-1\right)_\mathcal{B},
        \left(1,-2,1,2\right)_\mathcal{B},
        \left(0,1,-1,0\right)_\mathcal{B}
        \right\}
\]
Ahora, aplicando la definición:
\[
        v_\mathcal{B} = B_\mathcal{B} v_\mathcal{D} \hspace{2em} v_\mathcal{D} = B_\mathcal{B}^{-1} v_\mathcal{B}
\]
\subsection{Variedad Lineal}
Considerando un conjunto de vectores \(v \in \mathcal{V}\), llamamos \textbf{variedad lineal generada} por los vectores de \(v\) al conjunto de \textbf{CL} posibles de dichos vectores, se denota por \(L=\mathcal{L}\left\langle v_1, \cdots, v_n\right\rangle \). Es decir:
\[
        \mathcal{L}\left\langle v_1, \cdots, v_n\right\rangle = \left\{\sum_{i=1}^n\alpha_i v_i \hspace{.25em} : \hspace{.25em} \alpha_1,\cdots,\alpha_n \in \mathbf{R}^n\right\}
\]
\vspace{5em}
\par  Propiedades:
\begin{itemize}
        \item \(L\) es un subespacio vectorial de \(\mathcal{V}\).
        \item El conjunto de vectores \(v\) es un \textbf{Sg} de\(L\).
        \item Si eliminamos los vectores \textbf{LI} de \(v\), obtenemos una base de \(L\).
        \item Si \(L\) contiene a un solo vector, trabajamos con una recta; si tiene 2, con un plano; si tiene 3 con un espacio tridimensional, etc
        \item \(0 \leq \text{dim}\left(L\right) \leq \text{dim}\left(\mathcal{V}\right) \), por lo tanto:
              \begin{itemize}
                      \item \(\text{dim}\left(L\right) = 0 \Leftrightarrow L = \left\{\emptyset\right\} \)
                      \item \(\text{dim}\left(L\right) = \text{dim}\left(\mathcal{V}\right) \Leftrightarrow L = \mathcal{V} \)
              \end{itemize}
\end{itemize}
\subsubsection{Ecuaciones de Base}
Existen distintas formas de representar los vectores de una base \(\mathcal{B}\) en una \underline{variedad lineal} \(L\):
\begin{enumerate}
        \item \underline{\textbf{Ecuación Vectorial}}: \(x_\mathcal{B} = \sum_{i=1}^n \alpha_i v_{i\mathcal{B}}\)
        \item \underline{\textbf{Ecuación Paramétrica}}: Es el sistema de ecuaciones que obtenemos.
        \item \underline{\textbf{Ecuaciones Implícitas}}: Expresandose como un sistema de ecuaciones o imponiendo que el número de ecuaciones sea igual a \(\text{dim}\left(\mathcal{V}\right) = \text{dim}\left(\mathcal{L}\right) \), es decir, \(\text{Rango}\left(u_{1\mathcal{B}} \left\lvert \cdots \right\rvert  u_{n\mathcal{B}}\right)  =
              \text{Rango}\left(u_{1\mathcal{B}} \left\lvert \cdots \right\rvert  u_{n\mathcal{B}| x_\mathcal{B}}\right)
              \)
\end{enumerate}
\[
        \begin{pmatrix}
                x \\
                y \\
                z
        \end{pmatrix}
        =
        \alpha_1 \begin{pmatrix}
                1 \\
                0 \\
                -2
        \end{pmatrix} + \alpha_2 \begin{pmatrix}
                0  \\
                -1 \\
                3
        \end{pmatrix}
        \rightarrow_{1,2}
        \spalignsys{
                x = \alpha_1;
                y = \alpha_2;
                z = 3\alpha_2 - 2\alpha_1;
        }
        \rightarrow_{2,3} 2x + 3y +z = 0
\]
Ejemplo:
\[\mathcal{B} = \left\{v_1,\cdots, v_5\right\} \hspace{1em} L = \mathcal{L}\left\langle u_1,\cdots, u_4\right\rangle  \]
\[
        \spalignsys{
                u_1 = v_1 - v_2 + 2v_3 + v_4;
                u_2= v_2 - 2v_3 + v_4 + 2v_5;
                u_3= v_2 + 6v_3 + 3v_4 -4v_5;
                u_4= -v1 + 2v_2 + v_4 -v_5;
        } \Rightarrow
        \spalignsys{
                u_1 (1\text{,}-1\text{,}2\text{,}1\text{,}0);
                u_2 (0\text{,}1\text{,}-2\text{,}1\text{,}2);
                u_3= (0\text{,}1\text{,}6\text{,}3\text{,}-4);
                u_4= (-1\text{,}2\text{,}0\text{,}1\text{,}-1)}
\]
Si queremos obtener una base de \(L\) en \(B\), debemos sacar las ecuaciones implícitas de \(L\) en base \(B\), por lo que el rango entre los vectores de \(L\) y los de \(L \bigcup \left\{x_\mathcal{B}\right\} \) deben de ser iguales:
\[
        \begin{pmatrix}
                1  & 0  & 0  & -1 & | x_1 \\
                -1 & 1  & 1  & 2  & |x_2  \\
                2  & -2 & 6  & 0  & |x_3  \\
                1  & 1  & 3  & 1  & |x_4  \\
                0  & 2  & -4 & -1 & |x_5
        \end{pmatrix}
        \xrightarrow[\text{Usando Gauss}]{\text{Hacemos transformaciones elementales}}
        \begin{pmatrix}
                1 & 0 & 0 & -1 & x_1                      \\
                0 & 1 & 1 & 1  & x_1 + x_2                \\
                0 & 0 & 2 & 1  & -2x_1 - x_2 +x_4         \\
                0 & 0 & 0 & 0  & 8x_1 + 6x_2 + x_3 - 4x_4 \\
                0 & 0 & 0 & 0  & -8x_1 -5x_2 - 3x_4 + x_5
        \end{pmatrix}
\]
Vemos que el rango de \(L\) es 3, por lo que para que el rango de la matriz ampliada se iguale, obtenemos las ecuaciones de \(L\) en base \(\mathcal{B}\) :
\[
        \spalignsys{
                8x_1 + 6x_2 + x_3 - 4x_4 = 0;
                -8x_1 -5x_2 - 3x_4 + x_5 = 0;
        }
\]
Por otro lado debemos de convertir la matriz escalonada que obtuvimos antes, a una diagonal unitaria, para así obtener una base de \(L\), y como sabemos que el rango debe de ser 3, solo tendrá 3 variables esta base:
\[
        \begin{pmatrix}
                1 & 0 & 0 & -1          \\
                0 & 1 & 0 & \frac{1}{2} \\
                0 & 0 & 1 & \frac{1}{2} \\
                0 & 0 & 0 & 0           \\
                0 & 0 & 0 & 0
        \end{pmatrix} \equiv \left \{ \hat{u_1}, \hat{u_2}, \hat{u_3}, -\hat{u_1} + \frac{1}{2}(\hat{u_2} + \hat{u_3}) \right \}
\]
\subsection{Operaciones entre Espacios}
\begin{itemize}
        \item \underline{Intersección} \(L_1 \cap  L_2 \):  Obtenemos las ecuaciones implícitas de ambos espacios (juntando)
        \item \underline{Suma} \(L_1 +  L_2 \Rightarrow \mathcal{L}\left\langle v_1,\cdots,w_1,\cdots\right\rangle \):   Juntamos los dos espacios y sacamos los vectores \textbf{LI}.
        \item \underline{Suma Directa}\(L_1 \bigoplus  L_2 = L_1 + L_2 \Leftrightarrow L_1 \cap  L_2 = \left\{\emptyset\right\} \): Si la suma nos da como resultado el mismo espacio vectorial sobre el que trabajamos, \(\mathcal{V}\) entonces, ambos espacios contienen variables complementarias.
\end{itemize}
\subsection{Producto Escalar}
Lo denotamos como \(\left\langle u,w\right\rangle  \in \mathcal{V}\), con \(u_\mathcal{B}\) y \(w_\mathcal{B}\) siendo vectores no nulos. Por definición el producto escalar es:
\[
        \boxed{\left\langle u,w\right\rangle = \left\lVert u\right\rVert_2 \left\lVert w\right\rVert_2 \cos{\sigma} }
\]
Sin embargo, esta es la definición que nos sirve al estudiar la ortogonalidad, de forma general funciona de una forma distinta:
\[
        \boxed{\left\langle u,w\right\rangle =
                \begin{pmatrix}
                        u_1 & \cdots & u_n \\
                \end{pmatrix}_\mathcal{B}
                Q
                \begin{pmatrix}
                        w_1    \\
                        \vdots \\
                        w_n    \\
                \end{pmatrix}}_\mathcal{B}
\]
¿Qué es esta matriz \(Q\)? Es una matriz compuesta por todas las combinaciones posibles entre los vectores que conforman la base con la que trabajamos, de forma que sus valores son el producto escalar entre 2 vectores cualquiera de esa base:
\[
        Q = \begin{pmatrix}
                \left\langle v_1,v_1\right\rangle & \cdots & \left\langle v_1,v_n\right\rangle  \\
                \vdots                            & \ddots & \vdots                             \\
                \cdots                            & \cdots & \left\langle v_n, v_n\right\rangle
        \end{pmatrix}
\]
Aquí \(v \in \mathcal{B}\). Por lo tanto podemos describir el producto escalar de la siguiente forma:
\[
        \boxed{\left\langle u,w\right\rangle = u^t Q w = \sum^n_{i=1} \sum_{j=1}^n u_i w_j \left\langle v_i,v_j\right\rangle}
\]
\subsubsection{Ortogonalidad}
Decimos  que dos vectores son ortogonales cuando su producto escalar es cero, son perpendiculares. \textbf{También sabemos que el producto escalar es conmutativo} y
\subsection{Ortonormalidad}
Decimos que tenemos una base ortonormal cuando:
\begin{itemize}
        \item \textbf{El producto escalar de un vector por si mismo vale \(1\)}
        \item \textbf{El producto escalar por el resto de combinaciones de vectores, vale cero, es decir, son ortonales}.
\end{itemize}
\subsubsection{Ortonormalidad de Gram-Schmidt}
Para obtener un conjunto de vectores ortonormales debemos de obtener una base ortonormal, y para esto aplicaremos el algoritmo de Gram-Schmidt:
\[
        \boxed{w_k = u_k - \sum^{k-1}_{n=1} w_n\frac{\left\langle u_k, w_n\right\rangle }{\left\langle w_n, w_n\right\rangle }}
\]
Siendo \(u_k\) el vector que conforma un espacio concreto y \(w_k\) vectores ortogonales entre si, de forma que para que sean ortonormales, crearemos un espacio tal que:
\[
        v = \left(\frac{w_1}{\left\lVert w_1\right\rVert }, \cdots, \frac{w_k}{\left\lVert w_k\right\rVert }\right)
\]
\subsubsection{Variedad Ortogonal}
Denotamos a aquellas variedades, tal que el producto escalar entre los vectores de dos bases, da siempre cero.