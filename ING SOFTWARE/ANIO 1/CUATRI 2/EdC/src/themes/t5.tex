\subsection{Aplicación Lineal}
Llamamos a una aplicación \(f\) entre dos espacios vectoriaes \(\mathcal{V}\), y \(\mathcal{W}\) es una regla que asigna a cada elemento de \(x\) de \(\mathcal{V}\) un elemento \(f(x)\) de \(\mathcal{W}\). Normalmente se escribe así \(f: \mathcal{V} \rightarrow \mathcal{W}\)
\par Llamaremos imagen de \(x\) por \(f\) a \(f(x)\), siendo el argumento la \textbf{preimagen}.
\par Para que \(f\) sea una aplicación lineal u \textbf{homomorfismo} si:
\begin{itemize}
        \item \(f(u+v) = f(u) + f(v) \hspace{.5em} \forall u,v \in \mathcal{V}\)
        \item \(f(\alpha u) = \alpha f(u) \hspace{.5em} \forall \alpha \in \mathbb{R}, u \in \mathcal{V}\)
        \item Además, si cumple estas dos propiedades \(f(0) = 0\)
\end{itemize}
Terminamos diciendo, que toda aplicación lineal sobre su mismo espacio \(f: \mathcal{V} \rightarrow \mathcal{V}\), se le llama  \textbf{endomorfismo}. Veamos un ejemplo de aplicación lineal:
\[
        f: \mathbb{R}^2 \rightarrow \mathbb{R}^2 \hspace{2em} f(x,y) = (x^2, x + y)
\]
\[
        f(u + v) = f(u) + f(v) = f(x + x') + f(y + y') = (2x + 2x', x + y + x' + y') = (2x, x + y) + (2x', x' + y')
\]
\[
        f(\alpha u) = \alpha f(u) =  f(\alpha x, \alpha y) = (2\alpha x, \alpha x + \alpha y) = \alpha (2x, x +y)
\]
\(f(x,y)\) Es un endomorfismo
\subsubsection{Expresión Matricial}
Para manejar las aplicaciones de forma sencilla, obtendremos una expresión matricial en base a la base del espacio del que partimos y del que deseamos llegar. A grosso modo, tomamos las coordenadas en base \(\mathcal{B}\) de un vector genérico y hallamos su imagen usando las dos propiedades anteriores, y finalmente a este vector, le multiplicamos una matriz con las coordenadas en base \(\mathcal{D}\) en la otra base:
\[
        \boxed{f(X)_\mathcal{D} = A_\mathcal{D} X_\mathcal{B} = \begin{pmatrix}
                        f(v_1)_\mathcal{D} & \left | \cdots \right | & f(v_n)_\mathcal{D}
                \end{pmatrix}
                \begin{pmatrix}
                        x_1
                        \\
                        \vdots
                        \\
                        x_n
                \end{pmatrix}_\mathcal{B}}
\]
Ejemplo:
\[
        \mathcal{B} = \left\{v_1,v_2,v_3\right\} \in \mathbb{R}^3 \hspace{1em} \mathcal{D} = \left\{ w_1, w_2\right\} \in \mathbb{R}^2
\]
\[
        \spalignsys{
                f(v_1) = 2 w_1 - w_2 ;
                f(v_2) = w_1 + 4w_2  ;
                f(v_3) = -3w_1 + 2w_2;
        }
        \spalignsys{
                f(v_1) = (2\text{,} -1)_\mathcal{D};
                f(v_2) = (1\text{,} 4)_\mathcal{D};
                f(v_3) = (-3\text{,} 2)_\mathcal{D};
        }
\]
\[
        f(X)_\mathcal{D} = A = \begin{pmatrix}
                2  & 1 & -3 \\
                -1 & 4 & 2
        \end{pmatrix}
        X_\mathcal{B} = \begin{pmatrix}
                -3
                \\

                1 \\
                -2
        \end{pmatrix} \Rightarrow f(u) = w_1 + 3w_2
\]
\subsubsection{Operaciones}
Si \(f(X)\) la podemos denotar por \(M_fX\) entonces podemos realizar algunas operaciones sobre \(f\):
\begin{itemize}
        \item \textit{Multiplicar por un escalar}: \(\alpha f(X) = \alpha M_fX\)
        \item \textit{Multiplicar \(f\) con otra aplicación lineal}: \(f(X) + g(X) = (M_f + M_g)X\)
        \item \textit{Composición de otra aplicación lineal con \(f\)}: \(h(f(x)) = M_h M_f X\)
\end{itemize}
\subsubsection{Cambio de Base}
Consideremos dos nuevas bases \(\mathcal{E}\) y \(\mathcal{F}\) de los espacios \(\mathcal{V}\) y \(\mathcal{W}\) respectivamente, pertenecientes a la aplicación \(f: \mathcal{V} \rightarrow \mathcal{W}\) en las bases \(\mathcal{B}\) y \(\mathcal{C}\) \(f(X)_\mathcal{C} = A_\mathcal{C} X_\mathcal{B}\), Para hayar las matrices de cambio de base de  \(\mathcal{B} \rightarrow \mathcal{E}\) y \(\mathcal{C} \rightarrow \mathcal{F}\):
\[
        \boxed{\left.\begin{matrix}
                X_\mathcal{E} = P_\mathcal{E} X_\mathcal{B}
                \\
                X_\mathcal{F} = Q_\mathcal{F} X_\mathcal{C}
        \end{matrix}\right\}
        \Rightarrow f(X)_\mathcal{F} = Q_\mathcal{F} f(X)_\mathcal{C} = Q_\mathcal{F} A_\mathcal{C} X_\mathcal{B} = Q_\mathcal{F}A_\mathcal{C}P^{-1}_\mathcal{E}X_\mathcal{E}}
\]
\subsection{Imagen de una Variedad Lineal}
Sea \(f : \mathcal{V} \rightarrow \mathcal{W}\) una aplicación lineal de la matriz \(A\) en las bases \(\mathcal{B}\) y \(\mathcal{C}\), y  \(L\) una variedad lineal de \(\mathcal{V}\). La imagen de \(L\) por \(f\) se define como:
\[
        f(L) = \left\{f(v): v \in L\right\}
\]
Se llama \textbf{imagen de \(f\)} a \(\text{Im}f = f (\mathcal{V})\):
\begin{itemize}
        \item \(L = \mathcal{L}\left\langle u_1, \cdots, u_k\right\rangle \Rightarrow f(L) = \mathcal{L}\left\langle f(u_1), \cdots, f(u_k)\right\rangle  \)
        \item \(\text{dim}\left(\text{Im}f\right) = \text{Rango}\left(A\right)  \)
        \item Si la aplicación es sobreyectiva, \(\forall w \in \mathcal{W}\) tiene preimagen y no la tendrá si \(f(\mathcal{V}) \neq \mathcal{W}\), entonces \(\text{Im}f = \mathcal{W}\)
\end{itemize}
\subsubsection{Imagen Inversa}
La imagen inversa de \(L\) se define como:
\[
        f^{-1}(L)= \left\{v \in \mathcal{V} : f(v) \in L\right\}
\]
Siendo \(f\) una aplicación lineal de matriz \(A\) en las bases \(\mathcal{B}\) y \(\mathcal{C}\)
Y se le llama \textbf{kernel} de \(f\) a \(\ker f = f^{-1}({0})\). Para sacarla debemos de tener en cuenta ciertos detalles:
\begin{itemize}
        \item \(x \in f^{-1}(L) \Leftrightarrow f(x) \in L \Leftrightarrow Mf(X)_\mathcal{C}\Leftrightarrow MAX_\mathcal{B} = 0\)
        \item \(\ker f \equiv A X_\mathcal{B} = 0\) y \(\text{dim}\left(\ker f\right) = \text{dim}\left(\mathcal{V}\right) - \text{Rango}\left(A\right)   \)
        \item Si \(f\) es inyectiva el kernel de lafunción está vacío.
\end{itemize}
De los cuales llegamos a la siguiente conclusión:
\[
        L \equiv M X_\mathcal{C} \rightarrow f^{-1}(L) = MAX_\mathcal{B} = 0
\]