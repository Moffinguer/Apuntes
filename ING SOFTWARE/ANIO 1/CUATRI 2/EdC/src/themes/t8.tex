\subsection{Sustituciones}
\noindent Definimos una \textit{sustitución} \(\sigma \) a la aplicación de un objeto \(t\) a un término de nuestro Universo.
\par \noindent Los podemos definir de la siguiente forma:
\[
        \boxed{F\left [ x_1 / t_1, ..., x_n / t_n \right ]}
\]
\noindent Siendo \(F\) el término a evaluar, y \(x_n / t_n\) la sustitución del objeto \(x_n\) por el \(t_n\).
\par \noindent Un ejemplo es el siguiente:
\[\sigma = \left [ x / f(y),  y/b \right ]\]
\[A \equiv \forall_x (Q(x) \rightarrow R(x,y)) = \forall_x (Q(x) \rightarrow R(x,b))\]
\[B \equiv Q(x) \rightarrow \forall_x R(x,y) = Q(f(y)) \rightarrow R(x,b)\]
\noindent \textbf{Son fórmulas resultantes de una sustitución de \textit{ocurrencias libres}}.
\subsubsection{Sustituciones Libres}
\noindent Esta clase de sustituciones no introducen ninguna  ocurrencia nueva, de ninguna variable. Un ejemplo de una que no lo es:
\[\exists_x (x < y) =_{\left [ y/x \right ]} \exists_x(x<x)\]
\subsection{Reglas de Deducción Natural de Cuantificadores}
\noindent Vamos a aprovechar todas las reglas anteriores para esto, más unas pocas más:
\subsubsection{Regla del Cuantificador Universal Introducción}
\[
        \boxed{\frac{\boxed{\begin{matrix}
                                        x_o \hspace{2mm}\text{supuesto} \\
                                        \vdots                          \\
                                        F\left [ x/x \right ]
                                \end{matrix}}}{\forall_x F} \hspace{2mm} \forall i}
\]
\subsubsection{Reglas del Cuantificador Existencial Introducción}
\[
        \boxed{\frac{F\left [ x/t \right ]}{\exists_xF} \hspace{2mm} \exists i}
\]
\subsubsection{Regla del Cuantificador Universal Eliminación}
\[
        \boxed{\frac{\forall_x F}{F\left [ x/t \right ]} \hspace{2mm} \forall e
        }\]
\subsubsection{Reglas del Cuantificador Existencial Eliminación}
\[
        \boxed{\frac{\exists_x F\hspace{2mm}\boxed{\begin{matrix}
                                        x_o \hspace{2mm}\text{supuesto} \\
                                        \vdots                          \\
                                        G
                                \end{matrix}}}{G} \hspace{2mm} \exists e}
\]
\subsubsection{Ejemplos}
\[\left \{ \forall_x(P(x)\rightarrow Q(x)) \right \}
        \models \forall_x P(x) \rightarrow \forall_xQ(x)
\]
\[
        \boxed{\begin{matrix}
                        1)\hspace{2mm} \forall_x(P(x)\rightarrow Q(x)) \hspace{5mm} \text{Premisa}             \\
                        \hspace{2mm}2)\hspace{2mm} \forall_x P(x) \hspace{24mm} \text{Supuesto}                \\
                        3) x_0 \hspace{33mm} \text{Supuesto}                                                   \\
                        \hspace{-30mm}4)P(x_0) \hspace{5mm} \forall e \hspace{2mm} 2                           \\
                        \hspace{-13mm}5)P(x_0) \rightarrow Q(x_0)\hspace{5mm} \forall e \hspace{2mm} 1         \\
                        \hspace{-22mm}6)Q(x_0) \hspace{5mm} \rightarrow e \hspace{2mm} 4,5                     \\
                        \hspace{-24mm}7)\forall_x Q(x)\hspace{5mm} \forall i \hspace{2mm} 3,6                  \\
                        8) \forall_xP(x) \rightarrow \forall_xQ(x) \hspace{5mm} \rightarrow i \hspace{2mm} 2,7 \\
                \end{matrix}}
\]
\subsection{Reglas de Igualdad}