\subsection{Sintaxis}
\noindent Aprovecharemos toda la sintaxis dada en la \textit{Lógica Proposicional} más nuevas formas de representar el conocimiento, siendo más parecidas al lenguaje natural.
\subsubsection{Nuevos Elementos}
\noindent Representaremos nuestros entornos con 4 términos posibles:
\begin{itemize}
        \item \textbf{Constantes}: Objetos de nuestro entorno, cuyo valor es fijo. Ej: \((a,b,c,...\)
        \item \textbf{Variables}: Objetos de nuestro entorno los cuales pueden representar cualquier objeto de nuestro dominio del problema. Ej: \(x,y,z,...\)
        \item \textbf{Predicados}: Un tipo de expresión, que recibirá de 0 a N parámetros objeto, indicado por su \textit{aridad}, y el cual devolverá o verdadero o falso. Se representan con la letra mayúscula
        \item \textbf{Funciones}: Expresiones las cuales reciben objetos y devuelven otro objeto. Se representan con la letra minúscula.
\end{itemize}
\subsubsection{Nuevos Símbolos}
\begin{itemize}
        \item \(\forall\): Es un cuantificador universal, que indica que verifica la condición para cualquier objeto del dominio del problema.
        \item \(\exists\): Es un cuantificador existencial, que indica que la condición se verifica para al menos 1 elemento de nuestro dominio del problema.
        \item \(=\): Es un operador que compara dos objetos.
\end{itemize}
\subsection{Variables Libres o Ligadas}
\noindent Denominamos a estas variables a aquellas tales que dada una expresión, serán \textbf{libres} si aparece fuera de un operador cuantificador una vez, y \textbf{ligadas} si aparece dentro de uno.
\[\forall_x (P(x) \rightarrow R(x,y))\]
\noindent \(x\) es ligada mientras que \(y\) es libre.

\subsection{Estructura del Lenguaje}
\noindent Una estructura del lenguaje es un par \textbf{Universo}, \textbf{Interpretación}, lo representamos así:
\[\mathbb{I} = (U, \mathrm{I})\]
\begin{itemize}
        \item Denominamos \textbf{Universo} al conjunto de objetos que conforman nuestro dominio del problema.
        \item Denominamos \textbf{Interpretación} al conjunto de constantes, funciones y predicados que usaremos sobre el Universo.
\end{itemize}
\[
        \mathbb{I} = (\left \{ u,v,w \right \}, \left \{ (b,c),(P/1,Q/2),(f/2) \right \})
\]
\noindent Con estas dos herramientas, Universo e Interpretación, podremos obtener modelos de una expresión aplicando las interpretaciones a cada operador de la expresión, usando como objetos del problema los objetos del universo.
\par\noindent Diremos que una \textbf{estructura} e \textbf{interpretación} es una realización de \(F\) tal que la expresión tiene modelo con esa interpretación.
\par \noindent La \textbf{estructura} es un modelo de \(F\) cuando para toda interpretación de la estructura, \(F\) es un modelo.