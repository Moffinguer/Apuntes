\subsection{Pseudosolución de un Sistema Incompatible}
Este problema se conoce como \textbf{problema de mínimos cuadrados}, y se usa cuando tenemos un sistema de ecuaciones tal que no existe solución, de forma que buscamos un \(\hat{x}\) que se aproxime a lo que deseamos:
\[
        \boxed{\hat{x} = \left(A^tA\right)^{-1}At b }
        \boxed{\varepsilon = \left\lVert b - A \hat{x}\right\rVert_2}\hspace{1em} \text{Error de aproximación}
\]
\subsection{Transformaciones}
\subsubsection{Pseudosolución}
Podemos aplicar Housefolder para calcular pseudosoluciones de estos sistemas, usando el mismo algoritmo.
\par El error viene denotado por la \textit{norma euclídea} de la pseudopsolución.
\subsection{Recta de Regresión}
Dado un conjunto de puntos \(P_1(x_1, y_1), \cdots, P_n(x_n, y_n)\) buscamos la recta que mejor se adecúe a estos puntos, para esto calculamos la pseudosolución de \(A \binom{m}{n} b\) que equivale a:
\[
        \boxed{
                \begin{pmatrix}
                        x_1    & 1 \\
                        \vdots & 1 \\
                        x_n    & 1
                \end{pmatrix}
                \begin{pmatrix}
                        m
                        \\
                        n
                \end{pmatrix}
                =
                \begin{pmatrix}
                        y_1
                        \\
                        \vdots
                        \\
                        y_n
                \end{pmatrix}
        }\]