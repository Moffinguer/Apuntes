\subsection{Dificultades en el Despliegue}
\noindent Comparando una máquina virtual con un socket, podemos observar un problema fundamental: \textbf{una máquina virtual tiene que virtualizar desde el hardware hasta sus aplicaciones, mientras que un socket solo las aplicaciones}.
\par\noindent Usar varias máquinas virtuales es una solución, pero evidentemente no es factible, ocuparían mucho más espacio.
\subsection{Contenedores: Sockets}
\noindent Guardan en un repositorio las aplicaciones que necesite, y las máquinas que los vayan a ejecutar, tendrán programas capaces de desplegarlos, cada uno de forma aislada.
\par\noindent \textit{Docker}, \textit{Podman} o \textit{Linux Containers} son opciones viables.
\subsection{Escalabilidad y Tolerancia a Fallos}
\subsubsection{Escalabilidad}
\noindent Propiedad que indica la capacidad que tiene el sistema de ampliarse o decrementarse sin muchos cambios. Solo podemos hacerlo, al \textit{replicar aplicaciones sin estado}, que no tengan datos o si lo están, estén guardados de forma externa.
\subsubsection{Tolerancia a Fallos}
\noindent Propiedad que indica la robustez que tiene un sistema a cambios, sin posibilidad a errores. Si podemos desplegar un contenedor, sin romper el resto, será que tenemos un sistema robusto.
\subsection{Servicios en la Nube}
\noindent Hoy en día hay distintos tipos de implementaciones, \textbf{sobre hardware}, \textbf{hardware como servicio}, \textbf{infraetructura como servicio}, \textbf{plataforma como servicio} o \textbf{software como servicio}.