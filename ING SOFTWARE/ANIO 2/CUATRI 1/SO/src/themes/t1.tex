\subsection{Sistema Operativo}
El \textit{Sistema Operativo} es un software que facilita el uso del sistema informático, ``SI'', enlazando al usuario con el hardware, mediante las \textbf{interfaces entre las aplicaciones y el sistema} y las \textbf{interfaces entre el usuario y el sistema}.
\subsubsection{Kernel}
El \underline{kernel} o también llamado ``núcleo del sistema'', es una capa situada ``por encima'', un nivel de abstracción intermedio entre el \textit{hardware y las herramientas} que hacen la función de \textit{interfaces}, estas herramientas son de 3 tipos:
\begin{itemize}
        \item \textbf{Herramientas del Sistema}: Intérprete de comandos, administración del sistema, CLI ...
        \item \textbf{Herramientas de Desarrollo}: Compiladores, depuradores, IDE`s ...
        \item \textbf{Aplicaciones}: Navegador, juegos ...
\end{itemize}
\textbf{Linux no es un SO, sino un Kernel}
\subsection{Procesos e Hilos}
Los \textit{procesos} son objetos del ``SO'' que encapsulan un número determinado de \textit{hilos} en un espacio de memoria, y cuenta con recursos de \textbf{hardware y sofware}.\par
Gracias al \textit{kernel}, creamos una capa de abstracción, por la cual definimos un espacio de memoria propio para cada proceso, que no puede ser usado o visto por otro proceso, ya que cada espacio de memoria está definido para un solo proceso. \par \vspace{.5cm}
Es decir, \textbf{los procesos están incomunicados entre si}, sin embargo la cooperación entre procesos se debe al propio kernel, que proporciona herramientas, y además un proceso es capaz de crear otros procesos y acceder al sistema, unicamente solicitando servicios.
\par \vspace{.5cm}  \textbf{En Windows, el espacio estandar que se reserva por cada proceso es de 2Gi}
\subsubsection{Multiprogramación}
Considerando que tenemos varios procesos cargados, y que alguno está bloqueando al sistema (\textit{requiere una cantidad de tiempo muy superior al tiempo que puede cargar instrucciones el procesador}), podemos aprovechar la existencia de varios procesadores y ejecutar varios de estos procesos en cada procesador, aumentando la velocidad.
\par \vspace{.5cm}  Podemos suponer que cada procesador ejecutará sus procesos de forma independiente al resto, sin embargo en la realidad, usaremos un \textbf{planificador}, \textit{que en vez de planificar procesos, planifica hilos}, que liberará un procesador para que otro ``trabaje'', para esto el planificador usará algoritmos de planificación tales como el \textit{FIFO}, \textit{LIFO} ...
\par  Estos algoritmos, trabajan siempre bajo la misma premisa:
\begin{enumerate}
        \item El proceso está \textbf{preparado}, este está buscando un procesador para poder ejecutarse y se mantendrá en esta etapa mientras el planificador lo indique.
        \item Cuando encuentra un procesador, el proceso se pone en \textbf{activo}, y se ejecuta.
        \item Al terminar el proceso, se queda \textbf{bloqueado}, deja de consumir tiempo del procesador y se mantendrá así hasta que se requiera volver a usarlo y volverá a estar \textbf{preparado}
        \item De forma intermedia, podemos \textbf{apropiarnos} del procesador en medio de la ejecución del proceso, lo que lo llevará a estar \textbf{preparado}
\end{enumerate}
El tiempo que tarda un proceso, no tiene relación con el tiempo que consume del procesador, por lo que las operaciones que dependan del sistema, se harán a través de herramientas del sistema.
\subsubsection{Hilos}
Denominamos a los \textit{hilos} como secuencias de procesos que comparten el mismo espacio de memoria. \par  Estos nacen debido a la \textit{paralelización} de procesos, cuando tenemos varios procesadores podemos intentar ejecutar varios procesos a la vez, sin embargo si estos procesos dependen del estado de los otros, al no poder compartir recursos entre ellos, usamos un espacio de memoria reservado para un conjunto de procesos, de forma que ahora pueden comunicarse entre si. Solo no es rentable cuando se ejecutan operaciones de cálculo.
\subsection{Conceptos básicos sobre Hardware}
\subsubsection{Organización básica de un Ordenador}
El \textit{ordenador} es un conjunto de procesadores, unidos a una memoria y a diversos controladores, que a su vez están conectados a un bus de datos. \par  Cada procesador accede a su parte de la memoria, \textbf{no pueden leerse y escribir a la vez}, pero al resto de la memoria que no está asignada a los procesadores, se podrá acceder de esta forma simultaneamente.
\subsubsection{Interrupciones}
Es una alteración en la secuencia de ejecución de las instrucciones del procesador. Tiene distintas causas:
\begin{itemize}
        \item \textbf{Interrupción del \textit{Hardware}}: Como desconectar algún controlador.
        \item \textbf{Excepciones}
        \item \textbf{Ejecución de instrucciones de petición de interrupción}: Como \textit{INT} o \textit{TRAP}
\end{itemize}
Las \textit{interrupciones} las tratamos de la siguiente forma:
\begin{enumerate}
        \item Termina de ejecutar la CPU la instrucción actual.
        \item Almacenamos el estado del procesador en un registro.
        \item La CPU pasa a modo \textit{supervisor}
        \item Determinamos la dirección de una \textit{Subrutina de Servicios de Interrupción}, ``SSI'', que es una tabla de vectores que almacena codigos de interrupciones.
        \item Retornamos el valor de la tabla de SSI y restauramos el estado anterior.
        \item Continuamos con la ejecución del proceso.
\end{enumerate}
\subsubsection{Excepciones}
No son más que errores que suceden durante la ejecución de una instrucción.
\subsubsection{Modos de Ejecución de un procesador}
El procesador ejecutará una serie de \textit{instrucciones}, operaciones aritméticas; lógicas o de movimiento de datos.
\par \vspace{.2cm}  Podemos diferencias dos tipos de modo de ejecución:
\begin{itemize}
        \item \textbf{Supervisor}: Con este modo, el procesador puede ejecutar cualquier instrucción. Para pasar de \textit{supervisor} a \textit{usuario}, no es necesaria cumplir ninguna condición.
        \item \textbf{Usuario}: Solo puede ejecutar las \textit{instrucciones no privilegiadas}, aquellas que no accedan a recursos del \textit{hardware} ni recursos del \textit{software} externos al actual. Para acceder al modo \textit{supervisor}, es necesaria la ejecución de una interrupción, y esto ocurrirá \underline{si y solo si, cuando el proceso que quiera ejecutar pertenezca al \textit{kernel}}.
\end{itemize}
\subsection{Memoria}
La memoria funciona como un array de bits, guardando en direcciones consecutivas, podemos acceder a través de un índice que empieza desde el 0, y puede transmitir en potencias de 2 hasta 8 bytes.
\subsubsection{Paginación}
Es una técnica que se implementa por Hardware, a través del ``MLU'' que divide las direcciones lógicas en partes, a través de un \textit{traductor} que accede a una tabla \textit{de paginación}.
\par \vspace{.2cm} La memoria se reordena al usar este traductor, dividiendo la memoria en dos partes:
\begin{itemize}
        \item \textbf{Página} pasa a ser \textbf{marco}, que almacena la dirección física, e indica la dirección donde se guarda la información.
        \item \textbf{Desplazamiento}, se mantiene igual pero solo puede tener un tamaño de \(2^k\)bits.
\end{itemize}
\subsubsection{Mapa de Bits}
Esto es la tabla de traducciones, que asocia el contenido de cada marco con una página. Dentro de este mapa existen ciertos bits, que informan sobre el estado y contenido de este mapa. Entre estos destacamos el bit \(P\) que si vale cero, indica que el marco no está asignado. Se usa para poder usar un proceso sin necesidad de que tenga todos los marcos cargados, a lo que denominamos \textbf{Memoria Virtual}.
\subsubsection{Memoria Virtual}
Por defecto, todos los marcos no están cargados, lo que lanzará una excepción y enviará la página a un marco libre. Si no hay marcos libres, se ejecutará un algoritmo para verificar qué pagina no está en uso, a futuro, y la cargará en este nuevo espacio.
\subsection{Acceso a Dispositivos}
Para los SO, los dispositivos no son más que cajas negras que pueden transmitir información, reciben órdenes e informan sobre su estado.
\par  Existen los \textit{controladores}, interfaces electrónicas del dispositivo, que controla el funcionamiento y recibe órdenes del procesador. Podemos crearlos específicos para un dispositivo.
\subsubsection{Adaptadores}
Son dispositivos que sirven como interfaces para comunicar un dispositivo con otro. Esta interfaz proporciona:
\begin{itemize}
        \item \textbf{Características mecánicas}: Como el número de pines.
        \item \textbf{Características eléctricas}: Como la velocidad o el número de bits.
        \item \textbf{Protocolos}
\end{itemize}
Podemos acceder a estos dispositivos a través de un mapeo de de los registros, buffers de lectura/escritura, que lanzaran interrupciones acorde al número de dispositivos que haya. De esta forma informan de los cambios de estado a través de:
\begin{itemize}
        \item \textit{Moficando el valor del registro}
        \item \textit{Con interrupciones}
\end{itemize}
Y se gestionen mediante sondeos, que no deben de hacerse con procesadores, y como ya hemos dicho, interrupciones.
\subsubsection{Arquitectura I/O}
\begin{itemize}
        \item \textbf{Driver}: Es un componente del Kernel, que controla dispositivos.
        \item \textbf{Sistema de Archivos}: Crea una capa de abstracción de un array de sectores, que permite el uso de archivos en cualquier lugar.
        \item \textbf{Biblioteca de Funciones}: Almacena API`s y pequeños programas concretos.
\end{itemize}
\subsection{Unidades de Almacenamiento}
\subsubsection{Unidades Magnéticas}
A través de un adaptador interfaz ``SATA'', ``USB'', ``SAS''... se conecta a un adaptador controlador, que se conectará a la RAM, volatil, y a un almacenamiento por placas magnéticas, permantente.
\par  Solo transmite información por sectores (dividido en placas ferromagnéticas y pistas concéntricas, cada cara puede leerse y escribirse), y es muy lento.
\[
        \boxed{T_a = T_m + nT_c + T_h + T_r + T_x}
\]
Siendo:
\begin{itemize}
        \item \(T_m\) es el tiempo de arranque.
        \item \(nT_c\) es el número de pistas a saltar por un factor de tiempo medio para saltar entre cada pista.
        \item \(T_h\) es un factor de inercia.
        \item \(T_r\) tiempo de rotación.
        \item \(T_x\) tiempo de lectura por sector.
\end{itemize}
Existen ciertas optimizaciones como por ejemplo, \textit{transferir bloques consequitivos}, reduciendo el tiempo de lectura notablemente, \textit{leer los sectores en el orden que pasan} o \textit{reservar bloques en caché}.
\subsubsection{SSD}
Sustituye al disco magnético, en vez de usar placas magnéticas usa un array de memorias flash.
\par  Al hacer operaciones de escritura o lectura, debemos de considerar un factor, cada 100.000 borrados la celda flash debería de modificarse, porque no borra completamente su contenido. Y ya que al borrar información, debemos de borrar una página de memorias flash, debemos de tener en cuenta que los datos persistentes que no deberían de borrarse, se almacenan en un buffer temporal y se vuelven a escribir. Se usan ciertas optimizaciones para evitar tener que cambiarlas continuamente:
\begin{itemize}
        \item \textbf{Wear Leveling}: Hay sectores que se escriben con mucha frecuencia, por lo que usando un Wear Leveling \textit{dinámico}, establecemos una relación entre las direcciones físicas con las lógicas, de forma que no reescribe en las direcciones lógicas directamente sino en las físicas, evitando escrituras masívas.\par  Por otro lado, el \textit{estático}, hace lo mismo, pero reestructurando los sectores, los mueve.
        \item \textbf{Trim}: Un comando implantado en el Kernel (Linux 2, Windows XP, ...) que resuelve el problema de tener que borrar y escribir un sector inmediatamente, haciendolos irrelevante, por lo que si son borrados no ocurre nada, y la actualización y borrado son más rapidos.
\end{itemize}
\subsection{Interfaz de Usuario}
Podemos catalogarlos en dos tipos:
\begin{itemize}
        \item \textbf{Terminales Externos}: A través de un emulador, nos conectamos a otro dispositivo.
        \item \textbf{Consolas}: Dispositivos que se conectan directamente con la máquina y permiten su interacción directa con él. Se conforma por 3 componentes independientes: \textit{Pantalla}, \textit{teclado}, \textit{ratón}.
\end{itemize}
\subsubsection{GPU}
Siendo este el \textit{adaptador de pantalla}, guarda información en memoria, el procesador o la memoria principal (siendo esta la más común), tras analizar el estado de la pantalla, \textit{memoria de vídeo}.\par  Dependiendo de como se use esta memoria, se puede analizar la información de dos formas (relacionadas con la calidad que se quiera presentar por pantalla):
\begin{itemize}
        \item \textbf{Modo Texto}: Se guarda el color, brillo, opacidad... de los caracteres, a partir de un array de caracteres.
        \item \textbf{Modo gráfico}: Se escanea un array de pixeles, y se guarda información sobre el RGB, código que expresa colores.
\end{itemize}
\subsubsection{Teclado}
Posee integrado un microcontrolador capaz de exlorar el array de caracteres del teclado varias veces por segundo, almacenando información sobre el estado de las teclas (pulsadas o levantadas), guardando entonces eventos al teclear o usar atajos de teclado.
\subsubsection{Ratón}
Funciona igual que el teclado, posee un microcontrolador que genera un evento por cada click que se hace, y otro por cada vez que se cancela el click, se levanta el botón. \par  En vez de guardar información sobre el tipo de tecla, como el teclado, se guarda información del ángulo de un vector posición, que se mide en ``Mickies''.
\subsection{Arranque del Sistema}
Todos los componentes ``hardware'' del sistema se deben de inicializar en el arranque, y siguen el mismo procedimiento:
\begin{enumerate}
        \item El procesador genera un proceso de inicialización que se transfiere a una dirección de memoria determinada.
        \item Si el sistema operativo, no se encuentra en ROM, llama al \textit{cargador de hardware}, sino, irse al punto 5.
        \item El \textit{firmware} o cargador de hardware (UEFI o BIOS), realiza un encendido mínimo y verifica controladores conectados, además de leer el \textit{cargador de software}.
        \item El cargador de software carga el kernel y le transfiere el control una vez cargado.
        \item Se inicializa el kernel y realiza ciertas funciones:
              \begin{enumerate}
                      \item Crea procesos de sesión y estructuras de datos.
                      \item Checkea el sistema.
                      \item Carga dispostivos en el kernel.
              \end{enumerate}
\end{enumerate}
\subsection{Terminología de Redes}
\begin{itemize}
        \item \textbf{Adaptador de red}: Dispositivo que permite conectar el ordenador a la red.
        \item \textbf{Dirección MAC}: Dirección única que permite identificar al adaptador de red en una red local.
        \item \textbf{Switch}: Dispositivo que permite conectar varios segmentos de red en un punto, permitiendo controlar el tráfico de red.
        \item \textbf{Router}: Dispositivo que es capaz de conectar varias redes.
        \item \textbf{Gateway}: Dispositivo que se usa para conectar redes con distintos protocolos.
        \item \textbf{Dirección IP}: Dirección que identifica a cada ordenador de una red. Existen dos tipos, \textit{IPv4} e \textit{IPv6}, pero trabajaremos con la primera. \par  Se usa el formato ``CIDR'', \(xxx\).\(xxx\).\(xxx\).\(xxx\)/\(n\), siendo \(xxx\) un número de 3 dígitos y \(n\) el número de bits usados para la máscara de red. Hay hasta \(32-n\) bits para los hosts, que deben de estar a cero. \par  Además podemos catalogar las redes IPv4 en 3 tipos:
              \begin{itemize}
                      \item  \textbf{Clase A}: 24 bits para los hosts, (10.0.0.0 \(\rightarrow\) 10.255.255.255).\item  \textbf{Clase B}: 20 bits para los hosts, (172.16.0.0 \(\rightarrow\) 172.31.255.255).\item \textbf{Clase C}: 16 bits para los hosts, (192.168.0.0 \(\rightarrow\) 192.168.255.255).
              \end{itemize}
\end{itemize}
\subsection{Conceptos Básicos de SSOO}
\subsubsection{Llamadas al Sistema}
No son más que peticiones que se hacen al núcleo del SO par aobtener algún servicio, a través de un proceso. Normalmente lo vemos representado como las API, conjunto de llamadas al sistema que ejecutan programas, que se pueden implementar a través de:
\begin{itemize}
        \item \textit{Interrupciones}: Se accede mediante una interrupción, que pasará el sistema a modo supervisor, y podremos acceder a una dirección de memoria en función del hardware. Finalmente se restaurá el estado previo a la llamada.
        \item \textit{Instrucciones Específicas}: No importa la dirección de entrada y se ejecutan directamente en modo supervisor, sin embargo debe de hacer todas las comprobaciones que haría el modo supervisor.
        \item \textit{Rutinas}: No son factibles, debido a que es complicado calcular la dirección de memoria que estarían las rutinas y no pueden pasar al modo supervisor.
\end{itemize}
\subsubsection{Usuarios}
No son más que personas autorizadas a usar el sistema, que determinarán que derechos tendrán sobre un proceso que acceda al sistema.\par  Se les identifica con un UID, y si se agrupan en grupos, con un GID
\subsubsection{Archivos}
Conjunto de información guardado, estructurado de forma jerárquica en directorios, que residen en dispositivos.
\subsubsection{Intérprete de Comandos}
Programa interactivo que lee comandos del usuario via terminal y lo traduce a llamadas del sistema.
\subsubsection{Interfaces Gráficas de Usuario}
Programa interactivo que realiza la misma función que un intérprete de comandos pero de forma gráfica, GUI.
\subsection{Modelos de Diseño}
\subsubsection{Modelo Monolítico}
Todo el sistema se encuentra alojado en un único espacio de memoria, que no desorganizado. Por lo que todos los procesos y estructuras de datos son accesibles y manipulables entre si. Usa un módulo que controla los servicios, llamado \textbf{despachador}, una única rutina que hace las llamadas. Linux es monolítico.
\subsubsection{Modelo en Estratos}
Capas separadas para generar una mayor capa de abstracción, resolviendo cada una tareas específicas. Entre el usuario y el Hardware distinguimos 4 capas, llendo de mayor a menor nivel de cercania al hardware, tenemos:
\begin{itemize}
        \item \textbf{Capa 0}:Proporciona Hardware Multiprogramado.\item \textbf{Capa 1}: Dota a cada proceso de su propio estado de memoria.\item \textbf{Capa 2}: Dota a cada proceso de su propia consola virtual.\item \textbf{Capa 3}: Cada proceso es capaz de usar dispositivos de entrada y salida sobre otros.
\end{itemize}
Vemos que es facil de depurar pero perdemos la compartición de memoria.
\subsubsection{Modelo Micronúcleo}
Dividimos el nucleo en sectores, manteniendo las interrupciones y la comunicación entre procesos en el \textit{microkernel} y el resto de funcionalidades en modulos externos, menos aquellas funcionalidades que requieran de multiprogramación.
\par  Ofrece una gran robustez y facilidad de depuración, pero es extremadamente lento. El único SO, que ha logrado implementar este modelo es el descontinuado Minix.
\subsubsection{Ejemplos}
\begin{itemize}
        \item En el caso de \textbf{Linux}, al ser Monolítico, requiere de su gran comunidad para mantenerse, y su núcleo carga los módulos necesarios para realizar las tareas, por lo que es extremadamente rápido.
        \item \textbf{Windows} es también Monolítico, y separa los procesos de aplicaciones del propio kernel, sin embargo manteniendo las aplicaciones de usuario en el lado  de los procesos de servicio.
\end{itemize}