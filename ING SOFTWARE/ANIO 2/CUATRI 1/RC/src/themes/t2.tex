\subsection{Creación de Aplicaciones en Red}
\noindent Crear una aplicación en red no es más que desarrollar un programa que se ejecutan en distintos dispositivos, \underline{sistemas finales}.
\par \noindent Estos programas se comunican a través de la red, como los navegadores.
\par \vspace{.3cm}
\noindent No debemos de ahcer programas para el núcleo de la red, ya que su finalidad es encaminar los paquetes de la forma más rápida y eficiente posible, por lo que no necesitamos ningún programa extra.
\subsection{Arquitecturas}
\subsubsection{Arquitectura Cliente-Servidor}
\noindent Tenemos evidentemente dos componentes:
\begin{itemize}
        \item \textbf{Servidor}: Siempre está encendido y posee una IP fija, un identificador único, ya que si varía nunca sabríamos la dirección del servidor. Agruparlos creamos \textit{granjas de servidores}
        \item \textbf{Cliente}: Se conectan con el \textit{servidor}. Los clientes no se comunican entre ellos directamente.
\end{itemize}
\subsubsection{Arquitectura Peer To Peer}
\noindent No están siempre encendidos, y sus sistemas finales se comunican entre si de forma aleatoria, debido a esto sus ip cambian cada vez que se comunican entre ellos.
\par \noindent Son escalables, pero no de muy facil mantenimiento.
\subsubsection{Arquitectura Híbrida}
\noindent Tenemos un servidor centralizado que nos permite conocer la IP del usuario de forma inmediata, conectandose a una conexión ``peer to peer''.
\subsection{Implementación}
\noindent Aprovechará los servicios de la capa de \textit{Transporte}.
\subsubsection{Definicion}
\noindent Esta capa define:
\begin{itemize}
        \item \textit{Mensaje a intercambiar}.
        \item \textit{Sintaxis del mensaje}: Número de Campos, espacios, etc...
        \item \textit{Semántica del mensaje}: Qué significa cada campo.
        \item \textit{Reglas sobre como responder y recibir los mensajes}, los ICP.
\end{itemize}
\subsubsection{Comunicación entre procesos}
\noindent Los procesos son programas que se ejecutan en un dispositivo e implementa un protocolo. Hay dos tipos de comunicación:
\begin{itemize}
        \item \textbf{Mensajes PDU}: Los equipos se comparten la información usando mensajes PDU.
        \item \textbf{Comunicación entre procesos}: Hay dos o más procesos que se ejecutan paralelamente en un equipo, los proporcionan el SO.
\end{itemize}
\noindent Y dos tipos de procesos:
\begin{itemize}
        \item \textbf{Cliente}: Proceso que inicia la comunicación.
        \item \textbf{Servidor}: Proceso que espera a ser contactado.
\end{itemize}
\subsection{Sockets}
\noindent Los sockets son los puntos de acceso a través de los cuales se accede al servicio, para que el nivel de aplicación pueda enviar mensajes al nivel de transporte, es decir, es un ``SAP''.
\par \vspace{.2cm }\noindent Cada protocolo de aplicación se identifica con un número de puerto, que es usado para identificar el proceso cliente y servidor.\par \noindent \textbf{Es decir, un socket queda identificado por una dirección IP y un número de puerto}.
\subsection{LocalHost}
\noindent Localhost, con una dirección IP particular, \(127.0.0.1\), aunque puede variar, permite identificar al dispositivo final del sistema, el que esté usando una persona en ese momento.\par \noindent Puede probar aplicaciones en red, para un dispositivo final, sin necesidad de estar conectado a la red, y además de comunicarse con procesos dentro de ese equipo sistema final, usando sockets.
\subsection{Perdida de Datos}
\noindent Dependiendo de la aplicación, podemos tolerar un porcentaje de perdida de información, un \textit{tiempo de retardo} corto para ser efectiva o una \textit{tasa de transferencia} con una pérdida minima.
\par \noindent Evidentemente debemos de mantener una seguridad en el envio del mensaje, lo que puede provocar, si se viola, una perdida de la información.
\subsection{Servicios}
\subsubsection{TCP}
\begin{itemize}
        \item \textbf{Orientado a Conexión}: Requiere de un acuerdo entre los procesos cliente y servidor antes de iniciar la transferencia.
        \item \textbf{Transporte Fiable}: Solo entre procesos \textit{emisor} y \textit{receptor}.
        \item \textbf{Control de Flujo}: El \textit{emisor} no satura al \textit{receptor}
        \item \textbf{Control de Congestión}: Se evita la creción de cuellos de botella, controlando un uso equitativo del \textit{ancho de banda}.
        \item No \textbf{provee} de sincronización, ancho de banda o seguridad.
\end{itemize}
\subsubsection{UDP}
\begin{itemize}
        \item \textbf{Transporte Ligero}: No está orientado a la conexión y no es seguro entre procesos entre \textit{emisor} y \textit{receptor}
        \item No \textbf{provee} de un acuerdo previo entre procesos, seguridad, sincronización o control del flujo. Se usa para telefonía movil.
\end{itemize}
\subsection{DNS}
\noindent Es un protocolo de la capa de \textit{aplicación} que se encarga de generar identificadores que se encargen de diferenciar un sitio de otro, usando una \textit{base de datos distribuida}, la cual permitirá el acceso rápido a través de una jerarquía, en función de la frecuencia de búsqueda de estas direcciones, es decir, usa alias para relacionar las direcciones de las páginas con las búsquedas.
\subsubsection{Funcionamiento}
\begin{itemize}
        \item Enviamos una ``DNS-PDU'' al servidor DNS, la cual la reciribirá si hay un receptor, a través de un canal no protegido, UDP.
        \item El servidor enviará una respuesta al cliente de vuelta, si no hubiera un receptor anteriormente, se enviará tantas veces como sea necesario
\end{itemize}
\noindent Dentro de sus funcionalidades encontramos, la traducción de IP a alias o la asociación de un alias o varios a una IP. Por desgracia no podemos hacerla centralizada, la base de datos DNS, ya que no permitiría su escabilidad y daría problemas en caso de caida.
\par \noindent Hemos dicho que usa UDP como medio de transporte, normalmente usando el puerto 53.
\subsubsection{Memoria Caché}
\noindent Al igual que un ordenador, los servidores DNS, guardan en memoria de acceso rápido direcciones y alias que se usen con frecuencia, de forma que no es necesario buscarlas nuevamente, ya que se encuentran en un acceso fácil. En caso de no encontrarla, mandan la petición a otros servidores por encima de este, hasta que alguno lo encuentre, o no haya más servidores, en cuyo caso, se realizará la búsqueda inmediatamente.
\subsection{HTTP y Web}
\subsubsection{Páginas Web}
\noindent Las páginas web están formadas por una fichero HTML, que contiene objetos referenciados a los elementos que componen el fichero HTML, tales como imágenes, archivos JS, etc...
\par \noindent Cuando enviemos al servidor una petición para recoger una página, recopilaremos tantos objetos como referencias haya en el HTML, más una más por el propio fichero HTML.
\par \noindent Aqui, un ejemplo de un fichero HTML:
\lstset{language=HTML,keywordstyle=\color{blue},
        commentstyle=\color{green},
        stringstyle=\color{cyan},basicstyle={\small\ttfamily}}
\begin{lstlisting}
<!DOCTYPE html>
<html>
        <head>
        <!-- Este es un comentario en la cabecera-->
                <title>Hello World<\title>
        <\head>
        <body>
        <!-- Este es un comentario en el cuerpo-->
                <h1>Hello World but in boddy<\h1>
                <img src="hola.jpg"\>
        <\body>
<\html>
\end{lstlisting}
\subsubsection{Protocolo HTTP}
\noindent Protocolo, \textit{HyperText Transfer Protocol} a nivel de aplicación que al contrario que DNS, usa TCP para mantener un canal seguro. Usa el puerto 80 para comunicarse con el servidor, y usa la arquitectura \textbf{Cliente-Servidor} para responder a las consultas de los que accedan a sus recursos.
\begin{itemize}
        \item \textbf{Cliente}:Navegador que solicita objetos a un servidor Web.
        \item \textbf{Servidor}: Dispositivo que envia objetos pedidos por el cliente.
\end{itemize}
\noindent Los servidores no guardan información de peticiones anteriores, son \textit{sin extado} y podemos dividir las conexiones en dos tipos:
\begin{itemize}
        \item \textit{No persistente}: El cliente solicitará una petición al servidor, este aceptará la petición en caso de encontrarse, tras un tiempo, tras encontrar el servidor, se enviará un \textbf{mensaje de petición} por el cliente, con la petición de obtener el objeto HTML principal, y lo intentará recibir. Tras esto realizará una petición por un objeto del HTML, se le denomina \textbf{mensajes de respuesta} a aquellos mensajes que contengan el resto de objetos de la página. Finalmente se cerrará. Estos pasos se repetirán mientras aún no haya objetos transmitidos.
              \[
                      \boxed{TR = 2 RTT + TTP}
              \]
              \noindent Siendo \(RTT\) como \textit{el tiempo de ida y vuelta} en una petición al servidor, el tiempo que se tarda en enviar y recibir un mensaje de petición, y \(TTO\) el tiempo de \textit{transmisión de Bytes por objeto}.
              \par \noindent Podemos observar que este sistema es más lento, no permite paralelización y cargará uno a uno todos los elementos del HTML.
        \item \textit{Persistente}: El servidor mantendrá abierta la conexión tras enviar la respuesta de haber encontrado el objeto HTML, y el tiempo total será la suma de \(RTT\) por cada objeto más lo que tarde en enviarse cada objeto, se le denominará a esto \textit{solicitud de descarga del objeto}.
              \[
                      \boxed{\sum^n_{i=0}RTT_n + K}
              \]
\end{itemize}
\noindent Los mensajes HTTP pueden ser:
\begin{itemize}
        \item \textbf{Peticiones}: Enviadas por el cliente, formadas por un texto ASCII, y transporta HTTP-PCI para solicitar los HTTP-UD. Pueden ser GET, POST, HEAD, PUT o DELETE.
        \item \textbf{Respuestas}: Enviada por el servidor, y transporta los HTTP-UD.
\end{itemize}
\subsection{Cookies}
\noindent No son más que elementos que guardan el estado del servidor, de forma que un servidor es capaz de identificar un usuario incluso si accede desde otros dispositivos.
\subsection{Servidor Proxy}
\subsection{Socket}