\subsection{Nuevas fórmulas}
\noindent A parte de las fórmulas \underline{\textbf{alfas} y \textbf{betas}} podremos usar dos nuevos tipos de fórmulas, que se usarán para eliminar los operadores.
\subsubsection{Fórmulas Gamma}
\par \noindent Los \textbf{términos básicos} son elementos ya existentes en el universo, por lo que usaremos esas fórmulas hasta que se nos acaben los elementos del universo.
\begin{table}[h]
        \begin{tabular}{|l|l|l|}
                \hline
                \(F\)              & \(F_1\)         & Explicacion                        \\ \hline
                \(\forall G\)      & \(G[x/t]\)      & Con \(t\) siendo un término básico \\ \hline
                \(\neg \exists G\) & \(\neg G[x/t]\) & Con \(t\) siendo un término básico \\ \hline
        \end{tabular}
\end{table}
\subsubsection{Fórmulas Delta}
\par \noindent Las \textbf{constantes} son nuevos elementos en el universo, por lo que las formulas resultantes no se repetirán.
\begin{table}[h]
        \begin{tabular}{|l|l|l|}
                \hline
                \(F\)              & \(F_1\)         & Explicacion                          \\ \hline
                \(\exists D\)      & \(D[x/a]\)      & Con \(a\) siendo una nueva constante \\ \hline
                \(\neg \forall D\) & \(\neg D[x/a]\) & Con \(a\) siendo una nueva constante \\ \hline
        \end{tabular}
\end{table}