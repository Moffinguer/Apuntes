\subsection{Introducción a la Ingeniería del Software}
\subsubsection{Qué es el Software}
\noindent El software, es el conjunto de herramientas, creadas para solventar un problema concreto y que es automatizable, mediante herramientas digitales. Es decir, el \textbf{software} construye, crea y confecciona un proyecto con herramientas informáticas, de forma eficiente, que funcionará como una aplicación digital.\par
\noindent Podemos decir que el software es:
\begin{itemize}
        \item Intangible
        \item Se desarrolla
        \item Se puede quedar obsoleto
\end{itemize}
\subsubsection{Tipos de Software}
\noindent En función de la orientación que pueda tener, el software puede ser:
\begin{enumerate}
        \item Dominio de la aplicación
              \begin{enumerate}
                      \item Sistemas de Información
                      \item Telecomunicaciones
                      \item SO
                      \item Procesadores

              \end{enumerate}
        \item Tipo de Desarrollo
              \begin{enumerate}
                      \item A medida
                      \item Producto
              \end{enumerate}

        \item Tipo de Sistema
              \begin{enumerate}
                      \item Hardware
                      \item Software
                      \item Mixto
              \end{enumerate}
\end{enumerate}
\subsubsection{Evolución del Coste del Software}
\noindent A medida que avanzamos en el tiempo y en la calidad que presentamos, el coste aumenta. Tal que la creación de Interfaces gráficas, o subir a la nube, e incluso del uso de circuitos integrados especializados, aumentará el coste.
\subsubsection{Reportes CHAOS}
\noindent Estos reportes indican una estadística sobre los principales problemas que tiene la creación de software.
\par \noindent Clasifica al software en 3 categorías
\begin{itemize}
        \color{green}\item \color{black} Éxito: Finalizado dentro de plazo y presupuesto.
              \color{yellow}\item \color{black}
              Con problemas: Finalizado fuera de plazo, sin cumplir los requisitos.
              \color{red}\item \color{black}
              Fracaso: Cancelada.
\end{itemize}
\noindent Por estadística, todos los años se ve que los proyectos acaban \color{yellow}con problemas\color{black}, los siguen aquellos que \color{green}acaban exitosamente\color{black}, y finalmente los que \color{red}fallan \color{black} son una gran minoría.
\par \noindent También el índice de éxito aumenta con el uso de metodologías \textit{ágiles} respecto a las de \textit{cascada}, y con la compra de software ya creado y que solo requiere de mejoras.
\par \noindent Es decir:
\begin{itemize}
        \item \color{green}Los factores de éxito \color{black} son:
              \begin{enumerate}
                      \item Implicación de los usuarios
                      \item Apoyo de directivos
                      \item Requisitos claros
                      \item Planificación
                      \item Expectativas realistas
                      \item Proyectos pequeños
                      \item Personal competente
                      \item Vision y objetivos claros
              \end{enumerate}
        \item \color{yellow}Las causas de problemas \color{black} son:
              \begin{enumerate}
                      \item Falta de Información y requisitos, cambiantes.
                      \item Falta de apoyo directivo
                      \item Herramientas incorrectas y personal no competente.
                      \item Uso de nuevas tecnologías
                      \item Plazos y expectativas no realistas
              \end{enumerate}
        \item \color{red}Las causas de fracasos \color{black} son las opuestas a las de éxito.
\end{itemize}
\subsubsection{Pilares de la Ingeniería}
\begin{enumerate}
        \item Vocabulario: Términos que se usan en un ámbito en concreto.
        \item Tecnologías: Tecnologías y recursos usados en el campo.
        \item Herramientas: Conjunto de instrumentos para desempeñar un trabajo en concreto.
        \item Buenas prácticas: Conjunto de acciones que dan buen resultado en el campo.
        \item Metodologías: Conjunto de procedimientos bien definidos que generan buenos resultados.
\end{enumerate}
\subsubsection{Definición formal}
\noindent ``\textit{La aplicación inteligente de principios probados,
        técnicas, lenguajes y herramientas para la creación
        y mantenimiento, dentro de un coste razonable, de
        software que satisfaga las necesidades de los
        usuarios}''\par \noindent Es decir, conjunto de herramientas y metodologías que dentro de unos límites establecidos por un cliente, satisfaga las necesidades del cliente mismo de forma inteligente y eficiente.
\subsubsection{Proyecto}
\noindent Es un esfuerzo temporal que se lleva a cabo en un corto periodo de tiempo para crear un producto concreto y único. Pueden ser:
\begin{itemize}
        \item Productivos: Masivos.
        \item Públicos: Administración pública.
        \item Sociales: Cumplen una función social.
        \item De vida: Proyectos personales
        \item Científico.
\end{itemize}
\subsubsection{Etapas de un proyecto, ciclo de Deaming}
\noindent Encontramos 4 etapas, que se repiten indefinidamente:
\begin{itemize}
        \item Plan: Planificación y diseño
        \item Do: Generación de resultados, como prototipos
        \item Check: Testeo
        \item Act: Actuar en función de los testeos anteriores.
\end{itemize}
\subsubsection{Roles}
\noindent De mayor a menor puesto de importancia:
\begin{enumerate}
        \item Director de Proyecto: Responsable de la ejecución del proyecto con capacidad ejecutiva
              para tomar decisiones sobre el mismo de acuerdo con el cliente.
        \item Ingeniero de Requisitos: Responsable de interactuar con
              clientes y usuarios para obtener sus necesidades y de desarrollar y
              gestionar los requisitos.
        \item Equipo de Desarrollo: Conjunto de personas implicadas en el desarrollo del software
        \item Equipo de Calidad: Conjunto de personas responsables de la calidad de los productos
              obtenidos, tanto documentación como software.
        \item Cliente: Responsable de la financiación del proyecto con capacidad ejecutiva
              para tomar decisiones sobre el mismo.
        \item Usuario: Target del producto.
        \item Responsable TIC del Cliente: Responsable del entorno tecnológico del cliente.
\end{enumerate}
\subsubsection{Normas y Estándares}
\noindent El estándar relativo a los procesos de vida del software que usaremos será el ``\textit{ISO/IEC/IEEE 12207:2017}'', que incluirá la adquisición de software, productos, servicios y suministros además del desarrollo que abarque el software de una organización.
\par \noindent Usaremos ``\textit{CMMI-DEV (2010)}'' como un modelo para la mejora y evaluación de procesos para el desarrollo y mantenimiento de software. Tiene 5 niveles:
\begin{enumerate}
        \item[1)] Procesos inpredictibles y muy poco controlados.
        \item[2)] Proyectos reactivos.
        \item[3)] Proyectos afectados por la organización y es proactivo.
        \item[4)] Procesos medidos y controlados.
        \item[5)]    Se centra en la mejora continua.
\end{enumerate}
\subsubsection{Software como producto de ingeniería}
\noindent Se denomina \textit{entregables} al conjunto de productos que deben de desarrollarse y entregarse al cliente.
