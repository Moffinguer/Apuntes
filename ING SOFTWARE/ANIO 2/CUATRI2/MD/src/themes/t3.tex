\subsection{Notación Uniforme}
\subsubsection{Fórmulas Alpha}
\begin{table}[h]
        \begin{tabular}{|l|l|l|}
                \hline
                \(F\)                         & \(F_1\)                  & \(F_2\)                 \\ \hline
                \(A_1 \land  A_2\)            & \(A_1\)                  & \(A_2\)                 \\ \hline
                \(\neg(A_1 \rightarrow A_2)\) & \(A_1\)                  & \(\neg A_2\)            \\ \hline
                \(\neg(A_1 \lor A_2)\)        & \(\neg A_1\)             & \(\neg A_2\)            \\ \hline
                \(A_1 \leftrightarrow A_2\)   & \(A_1 \rightarrow  A_2\) & \(A_2 \rightarrow A_1\) \\ \hline
        \end{tabular}
\end{table}
\noindent Estas fórmulas no generarán nuevos conjuntos.
\par \noindent El conjunto de formulas atómicas resultantes genera una expresión del tipo: \(F \equiv F_1 \land F_2\)
\subsubsection{Fórmulas Beta}
\begin{table}[h]
        \begin{tabular}{|l|l|l|}
                \hline
                \(F\)                             & \(F_1\)                        & \(F_2\)                       \\ \hline
                \(B_1 \lor  B_2\)                 & \(B_1\)                        & \(B_2\)                       \\ \hline
                \(B_1 \rightarrow B_2\)           & \(\neg B_1\)                   & \( B_2\)                      \\ \hline
                \(\neg(B_1 \land B_2)\)           & \(\neg B_1\)                   & \(\neg B_2\)                  \\ \hline
                \(\neg(B_1 \leftrightarrow B_2)\) & \(\neg(B_1 \rightarrow  B_2)\) & \(\neg(B_2 \rightarrow B_1)\) \\ \hline
        \end{tabular}
\end{table}
\noindent Estas fórmulas generarán nuevos conjuntos.
\par \noindent El conjunto de formulas atómicas resultantes genera una expresión del tipo: \(F \equiv F_1 \lor F_2\)
\subsubsection{Fórmulas interesantes}
\begin{itemize}
        \item \(F \equiv \neg \neg F\)
        \item \(\neg(A \land B) \equiv \neg A \lor \neg B\)
        \item \item \(\neg(A \lor B) \equiv \neg A \land \neg B\)
\end{itemize}
\subsection{Ejemplo}
\[\boxed{\left \{ \neg(\neg p \lor \neg q \rightarrow \neg(p\land r)) \right.\left.  \right \}}\]
\[\boxed{\left \{ \neg p \lor \neg q, \neg \neg (p\land r) \right.\left.  \right \}}\]
\[\boxed{\left \{ \neg p \lor \neg q, p, r \right.\left.  \right \}}\]
\[\boxed{\left \{ \neg p, p, r \right.\left.  \right \}} \hspace{5mm} \boxed{\left \{ \neg q, p, r \right.\left.  \right \}}\]
\[\hspace{20mm}\boxed{\perp } \hspace{10mm} \boxed{\text{Tendrá algún modelo}}\]