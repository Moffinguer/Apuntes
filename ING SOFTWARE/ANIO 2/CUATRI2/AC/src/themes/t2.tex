\subsection{Reglas}
\subsubsection{Reglas de la Conjunción}
\noindent \textbf{Introducción}:
\[
        \boxed{\frac{F\hspace{5mm}G}{F \land G} \land i}
\]
\noindent \textbf{Eliminación}:
\[\boxed{\frac{F \land G}{F}\land e_1 \hspace{.5cm} \frac{F \land G}{G}\land e_2}
\]
\subsubsection{Reglas de la Doble Negación}
\noindent \textbf{Introducción}:
\[
        \boxed{\frac{F}{\neg \neg F} \neg \neg i}
\]
\noindent \textbf{Eliminación}:
\[\boxed{\frac{\neg \neg F}{F} \neg \neg e}
\]
\subsubsection{Reglas del Condicional}
\noindent \textbf{Introducción}:
\[\boxed{\frac{\boxed{\begin{matrix}
                                        F      \\
                                        \cdots \\
                                        G
                                \end{matrix}}}{F \rightarrow G} \rightarrow i}\]
\noindent \textbf{Eliminación}:
\[\boxed{\frac{F \hspace{3mm}F\rightarrow G}{G}\rightarrow  e}
\]
\subsubsection{Reglas de la Disyunción}
\noindent \textbf{Introducción}:
\[
        \boxed{\frac{F}{F \lor G} \lor i_1 \hspace{5mm}\frac{G}{F \lor G} \lor i_2}
\]
\noindent \textbf{Eliminación}:
\[\boxed{\frac{F \lor G \hspace{2mm}
                        \boxed{\begin{matrix}
                                        F      \\
                                        \cdots \\
                                        H
                                \end{matrix}}
                        \hspace{2mm}
                        \boxed{\begin{matrix}
                                        G      \\
                                        \cdots \\
                                        H
                                \end{matrix}}}{H} \hspace{2mm}\lor e}
\]
\subsubsection{Reglas de la Negación}
\noindent \textbf{Introducción}, siendo \(\perp\) una contradicción:
\[
        \boxed{\frac{\boxed{\begin{matrix}
                                        F      \\
                                        \cdots \\
                                        \perp
                                \end{matrix}}}{\neg F} \hspace{2mm}\neg i}\]
\noindent \textbf{Eliminación}:
\[
        \boxed{\frac{\perp }{F} \perp e} \hspace{1cm} \boxed{\frac{F \hspace{2mm}\neg F}{\perp} \neg e}
\]
\subsubsection{Reglas del Bicondicional}
\noindent \textbf{Introducción}:
\[
        \boxed{\frac{F \rightarrow G \hspace{2mm} G \rightarrow F}{F \leftrightarrow G} \hspace{2mm} \leftrightarrow i}
\]
\noindent \textbf{Eliminación}:
\[\boxed{\frac{F \leftrightarrow G}{F \rightarrow G} \hspace{2mm} \leftrightarrow e_1 \hspace{5mm} \frac{F \leftrightarrow G}{G \rightarrow F} \hspace{2mm} \leftrightarrow e_2}
\]
\subsection{Reglas Derivadas}
\subsubsection{Regla derivada Modus Tollens}
\noindent Expresión:
\[
        \boxed{\frac{F\rightarrow G \hspace{3mm} \neg G}{\neg F} \hspace{5mm} MT}\]
\subsubsection{Regla de Reducción al Absurdo}
\noindent Expresión:
\[
        \boxed{\frac{\boxed{\begin{matrix}
                                        F      \\
                                        \cdots \\
                                        \perp
                                \end{matrix}}}{F} \hspace{2mm}RAA
        }
\]


