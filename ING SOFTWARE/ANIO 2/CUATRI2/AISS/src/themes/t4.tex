\subsection{Introducción}
\noindent Usando transformaciones buscamos crear expresiones tales que obtenemos funciones FNC, \textit{formas normales conjuntivas}, o FND, \textit{formas normales disyuntivas}.
\[
        \text{FNC} \equiv (F^1_1 \lor F^1_2 ...) \land ... (F^n_1 \lor F^n_2 ...)
\]
\[
        \text{FND} \equiv (F^1_1 \land F^1_2 ...) \lor ... (F^n_1 \land F^n_2 ...)
\]
\subsubsection{FNC}
\noindent Podemos afirmar que las expresiones de este tipo pueden demostrar si existe una tautología, solo en caso en el que haya un conjunto de disyunciones tales que den como resultado una incongruencia.
\par \noindent Si queremos saber si es una tautología, comprobamos que esta FND sea satisfacible:
\[
        \boxed{G = \text{FNC}(F) \Rightarrow \text{FND}(\neg F)}
\]
\subsubsection{FND}
\noindent Podemos afirmar que las expresiones de este tipo pueden demostrar si una expresión es satisfacible, solo y solo si podemos encontrar algún modelo, es decir haya alguna conjunción tal que sea contraria a otra.
\par \noindent Si queremos comprobar si es un insatisfacible, comprobamos que esta FNC sea una tautología
\[
        \boxed{G = \text{FND}(F) \Rightarrow \text{FNC}(\neg F)}
\]
\subsection{Expresiones a convertir}
\begin{itemize}
        \item \(A \rightarrow B \equiv \neg A\lor B\)
        \item \(A \leftrightarrow B \equiv (A \rightarrow B) \land (B \rightarrow A)\)
\end{itemize}